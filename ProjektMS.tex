% Options for packages loaded elsewhere
\PassOptionsToPackage{unicode}{hyperref}
\PassOptionsToPackage{hyphens}{url}
%
\documentclass[
]{article}
\usepackage{lmodern}
\usepackage{amssymb,amsmath}
\usepackage{ifxetex,ifluatex}
\ifnum 0\ifxetex 1\fi\ifluatex 1\fi=0 % if pdftex
  \usepackage[T1]{fontenc}
  \usepackage[utf8]{inputenc}
  \usepackage{textcomp} % provide euro and other symbols
\else % if luatex or xetex
  \usepackage{unicode-math}
  \defaultfontfeatures{Scale=MatchLowercase}
  \defaultfontfeatures[\rmfamily]{Ligatures=TeX,Scale=1}
\fi
% Use upquote if available, for straight quotes in verbatim environments
\IfFileExists{upquote.sty}{\usepackage{upquote}}{}
\IfFileExists{microtype.sty}{% use microtype if available
  \usepackage[]{microtype}
  \UseMicrotypeSet[protrusion]{basicmath} % disable protrusion for tt fonts
}{}
\makeatletter
\@ifundefined{KOMAClassName}{% if non-KOMA class
  \IfFileExists{parskip.sty}{%
    \usepackage{parskip}
  }{% else
    \setlength{\parindent}{0pt}
    \setlength{\parskip}{6pt plus 2pt minus 1pt}}
}{% if KOMA class
  \KOMAoptions{parskip=half}}
\makeatother
\usepackage{xcolor}
\IfFileExists{xurl.sty}{\usepackage{xurl}}{} % add URL line breaks if available
\IfFileExists{bookmark.sty}{\usepackage{bookmark}}{\usepackage{hyperref}}
\hypersetup{
  pdftitle={Projekt Metody Statystyczne - Dokumentacja},
  pdfauthor={Autorzy: Mateusz Adamczyk, Natalia Cheba, Tomasz Depczyński, Julia Kwapień, Anna Szkoda, Maciej Zajęcki},
  hidelinks,
  pdfcreator={LaTeX via pandoc}}
\urlstyle{same} % disable monospaced font for URLs
\usepackage[margin=1in]{geometry}
\usepackage{color}
\usepackage{fancyvrb}
\newcommand{\VerbBar}{|}
\newcommand{\VERB}{\Verb[commandchars=\\\{\}]}
\DefineVerbatimEnvironment{Highlighting}{Verbatim}{commandchars=\\\{\}}
% Add ',fontsize=\small' for more characters per line
\usepackage{framed}
\definecolor{shadecolor}{RGB}{248,248,248}
\newenvironment{Shaded}{\begin{snugshade}}{\end{snugshade}}
\newcommand{\AlertTok}[1]{\textcolor[rgb]{0.94,0.16,0.16}{#1}}
\newcommand{\AnnotationTok}[1]{\textcolor[rgb]{0.56,0.35,0.01}{\textbf{\textit{#1}}}}
\newcommand{\AttributeTok}[1]{\textcolor[rgb]{0.77,0.63,0.00}{#1}}
\newcommand{\BaseNTok}[1]{\textcolor[rgb]{0.00,0.00,0.81}{#1}}
\newcommand{\BuiltInTok}[1]{#1}
\newcommand{\CharTok}[1]{\textcolor[rgb]{0.31,0.60,0.02}{#1}}
\newcommand{\CommentTok}[1]{\textcolor[rgb]{0.56,0.35,0.01}{\textit{#1}}}
\newcommand{\CommentVarTok}[1]{\textcolor[rgb]{0.56,0.35,0.01}{\textbf{\textit{#1}}}}
\newcommand{\ConstantTok}[1]{\textcolor[rgb]{0.00,0.00,0.00}{#1}}
\newcommand{\ControlFlowTok}[1]{\textcolor[rgb]{0.13,0.29,0.53}{\textbf{#1}}}
\newcommand{\DataTypeTok}[1]{\textcolor[rgb]{0.13,0.29,0.53}{#1}}
\newcommand{\DecValTok}[1]{\textcolor[rgb]{0.00,0.00,0.81}{#1}}
\newcommand{\DocumentationTok}[1]{\textcolor[rgb]{0.56,0.35,0.01}{\textbf{\textit{#1}}}}
\newcommand{\ErrorTok}[1]{\textcolor[rgb]{0.64,0.00,0.00}{\textbf{#1}}}
\newcommand{\ExtensionTok}[1]{#1}
\newcommand{\FloatTok}[1]{\textcolor[rgb]{0.00,0.00,0.81}{#1}}
\newcommand{\FunctionTok}[1]{\textcolor[rgb]{0.00,0.00,0.00}{#1}}
\newcommand{\ImportTok}[1]{#1}
\newcommand{\InformationTok}[1]{\textcolor[rgb]{0.56,0.35,0.01}{\textbf{\textit{#1}}}}
\newcommand{\KeywordTok}[1]{\textcolor[rgb]{0.13,0.29,0.53}{\textbf{#1}}}
\newcommand{\NormalTok}[1]{#1}
\newcommand{\OperatorTok}[1]{\textcolor[rgb]{0.81,0.36,0.00}{\textbf{#1}}}
\newcommand{\OtherTok}[1]{\textcolor[rgb]{0.56,0.35,0.01}{#1}}
\newcommand{\PreprocessorTok}[1]{\textcolor[rgb]{0.56,0.35,0.01}{\textit{#1}}}
\newcommand{\RegionMarkerTok}[1]{#1}
\newcommand{\SpecialCharTok}[1]{\textcolor[rgb]{0.00,0.00,0.00}{#1}}
\newcommand{\SpecialStringTok}[1]{\textcolor[rgb]{0.31,0.60,0.02}{#1}}
\newcommand{\StringTok}[1]{\textcolor[rgb]{0.31,0.60,0.02}{#1}}
\newcommand{\VariableTok}[1]{\textcolor[rgb]{0.00,0.00,0.00}{#1}}
\newcommand{\VerbatimStringTok}[1]{\textcolor[rgb]{0.31,0.60,0.02}{#1}}
\newcommand{\WarningTok}[1]{\textcolor[rgb]{0.56,0.35,0.01}{\textbf{\textit{#1}}}}
\usepackage{graphicx,grffile}
\makeatletter
\def\maxwidth{\ifdim\Gin@nat@width>\linewidth\linewidth\else\Gin@nat@width\fi}
\def\maxheight{\ifdim\Gin@nat@height>\textheight\textheight\else\Gin@nat@height\fi}
\makeatother
% Scale images if necessary, so that they will not overflow the page
% margins by default, and it is still possible to overwrite the defaults
% using explicit options in \includegraphics[width, height, ...]{}
\setkeys{Gin}{width=\maxwidth,height=\maxheight,keepaspectratio}
% Set default figure placement to htbp
\makeatletter
\def\fps@figure{htbp}
\makeatother
\setlength{\emergencystretch}{3em} % prevent overfull lines
\providecommand{\tightlist}{%
  \setlength{\itemsep}{0pt}\setlength{\parskip}{0pt}}
\setcounter{secnumdepth}{-\maxdimen} % remove section numbering

\title{Projekt Metody Statystyczne - Dokumentacja}
\author{Autorzy: Mateusz Adamczyk, Natalia Cheba, Tomasz Depczyński, Julia
Kwapień, Anna Szkoda, Maciej Zajęcki}
\date{Data sprawozdania: 07-06-2020}

\begin{document}
\maketitle

\hypertarget{przygotowanie-danych-do-dalszej-analizy}{%
\subsection{Przygotowanie danych do dalszej
analizy}\label{przygotowanie-danych-do-dalszej-analizy}}

\begin{Shaded}
\begin{Highlighting}[]
\KeywordTok{library}\NormalTok{(tidyr)}
\CommentTok{#Zamiana nazwy kolumn na odpowiednie nazwy}
\KeywordTok{names}\NormalTok{(Autko)[}\DecValTok{1}\NormalTok{] <-}\StringTok{ "mpg"}
\KeywordTok{names}\NormalTok{(Autko)[}\DecValTok{2}\NormalTok{] <-}\StringTok{ "cylinders"}
\KeywordTok{names}\NormalTok{(Autko)[}\DecValTok{3}\NormalTok{] <-}\StringTok{ "displacement"}
\KeywordTok{names}\NormalTok{(Autko)[}\DecValTok{4}\NormalTok{] <-}\StringTok{ "horsepower"}
\KeywordTok{names}\NormalTok{(Autko)[}\DecValTok{5}\NormalTok{] <-}\StringTok{ "weight"}
\KeywordTok{names}\NormalTok{(Autko)[}\DecValTok{6}\NormalTok{] <-}\StringTok{ "acceleration"}
\KeywordTok{names}\NormalTok{(Autko)[}\DecValTok{7}\NormalTok{] <-}\StringTok{ "model_year"}
\KeywordTok{names}\NormalTok{(Autko)[}\DecValTok{8}\NormalTok{] <-}\StringTok{ "origin"}
\KeywordTok{names}\NormalTok{(Autko)[}\DecValTok{9}\NormalTok{] <-}\StringTok{ "car_name"}

\CommentTok{#Zmienienie na odpowiednie typy danych}
\NormalTok{Autko}\OperatorTok{$}\NormalTok{car_name<-}\KeywordTok{as.character}\NormalTok{(Autko}\OperatorTok{$}\NormalTok{car_name)}
\NormalTok{Autko}\OperatorTok{$}\NormalTok{cylinders =}\StringTok{ }\NormalTok{Autko}\OperatorTok{$}\NormalTok{cylinders }\OperatorTok\StringTok{ }\KeywordTok{factor}\NormalTok{(}\DataTypeTok{labels =} \KeywordTok{sort}\NormalTok{(}\KeywordTok{unique}\NormalTok{(Autko}\OperatorTok{$}\NormalTok{cylinders)))}
\NormalTok{Autko}\OperatorTok{$}\NormalTok{model_year =}\StringTok{ }\NormalTok{Autko}\OperatorTok{$}\NormalTok{model_year }\OperatorTok\StringTok{ }\KeywordTok{factor}\NormalTok{(}\DataTypeTok{labels =} \KeywordTok{sort}\NormalTok{(}\KeywordTok{unique}\NormalTok{(Autko}\OperatorTok{$}\NormalTok{model_year)))}
\NormalTok{Autko}\OperatorTok{$}\NormalTok{origin =}\StringTok{ }\NormalTok{Autko}\OperatorTok{$}\NormalTok{origin }\OperatorTok\StringTok{ }\KeywordTok{factor}\NormalTok{(}\DataTypeTok{labels =} \KeywordTok{sort}\NormalTok{(}\KeywordTok{unique}\NormalTok{(Autko}\OperatorTok{$}\NormalTok{origin)))}
\NormalTok{Autko}\OperatorTok{$}\NormalTok{horsepower<-}\KeywordTok{as.numeric}\NormalTok{(Autko}\OperatorTok{$}\NormalTok{horsepower)}
\end{Highlighting}
\end{Shaded}

\begin{verbatim}
## Warning: pojawiły się wartości NA na skutek przekształcenia
\end{verbatim}

\begin{Shaded}
\begin{Highlighting}[]
\CommentTok{#Oddzielenie marki i modelu }
\NormalTok{Autko <-}\StringTok{ }\KeywordTok{separate}\NormalTok{(Autko, }\DataTypeTok{col =} \KeywordTok{c}\NormalTok{(}\StringTok{"car_name"}\NormalTok{), }\DataTypeTok{into =} \KeywordTok{c}\NormalTok{(}\StringTok{"brand"}\NormalTok{, }\StringTok{"model"}\NormalTok{), }\DataTypeTok{sep =} \StringTok{" "}\NormalTok{, }\DataTypeTok{extra =} \StringTok{"merge"}\NormalTok{)}
\end{Highlighting}
\end{Shaded}

\begin{verbatim}
## Warning: Expected 2 pieces. Missing pieces filled with `NA` in 2 rows [151,
## 347].
\end{verbatim}

\begin{Shaded}
\begin{Highlighting}[]
\CommentTok{#Zmamiana "?" na wartosci N/A}
\NormalTok{Autko[Autko }\OperatorTok{==}\StringTok{ "?"}\NormalTok{] <-}\StringTok{ }\OtherTok{NA}
\CommentTok{#Zastępowanie wartości NULL średnią }
\NormalTok{Autko}\OperatorTok{$}\NormalTok{horsepower <-}\StringTok{ }\KeywordTok{ifelse}\NormalTok{(}\KeywordTok{is.na}\NormalTok{(Autko}\OperatorTok{$}\NormalTok{horsepower), }\KeywordTok{mean}\NormalTok{(Autko}\OperatorTok{$}\NormalTok{horsepower, }\DataTypeTok{na.rm=}\OtherTok{TRUE}\NormalTok{), Autko}\OperatorTok{$}\NormalTok{horsepower)}

\CommentTok{#Zamiana brandów na odpowiednie nazwy}
\NormalTok{Autko}\OperatorTok{$}\NormalTok{brand[Autko}\OperatorTok{$}\NormalTok{brand }\OperatorTok{==}\StringTok{ "chevroelt"}\NormalTok{] <-}\StringTok{ "chevrolet"}
\NormalTok{Autko}\OperatorTok{$}\NormalTok{brand[Autko}\OperatorTok{$}\NormalTok{brand }\OperatorTok{==}\StringTok{ "maxda"}\NormalTok{] <-}\StringTok{ "mazda"}
\NormalTok{Autko}\OperatorTok{$}\NormalTok{brand[Autko}\OperatorTok{$}\NormalTok{brand }\OperatorTok{==}\StringTok{ "vokswagen"}\NormalTok{] <-}\StringTok{ "volkswagen"}
\NormalTok{Autko}\OperatorTok{$}\NormalTok{brand[Autko}\OperatorTok{$}\NormalTok{brand }\OperatorTok{==}\StringTok{ "toyouta"}\NormalTok{] <-}\StringTok{ "toyota"}
\NormalTok{Autko}\OperatorTok{$}\NormalTok{brand[Autko}\OperatorTok{$}\NormalTok{brand }\OperatorTok{==}\StringTok{ "vw"}\NormalTok{] <-}\StringTok{ "volkswagen"}
\NormalTok{Autko}\OperatorTok{$}\NormalTok{brand[Autko}\OperatorTok{$}\NormalTok{brand }\OperatorTok{==}\StringTok{ "mercedes-benz"}\NormalTok{] <-}\StringTok{ "mercedes"}

\CommentTok{#Zamiana na factory}
\NormalTok{Autko}\OperatorTok{$}\NormalTok{brand =}\StringTok{ }\NormalTok{Autko}\OperatorTok{$}\NormalTok{brand }\OperatorTok
\StringTok{  }\KeywordTok{factor}\NormalTok{(}\DataTypeTok{labels =} \KeywordTok{sort}\NormalTok{(}\KeywordTok{unique}\NormalTok{(Autko}\OperatorTok{$}\NormalTok{brand)))}
\end{Highlighting}
\end{Shaded}

\hypertarget{statystyka}{%
\subsection{Statystyka}\label{statystyka}}

\begin{Shaded}
\begin{Highlighting}[]
\KeywordTok{library}\NormalTok{(e1071)}

\NormalTok{funkcja_dominanta <-}\StringTok{ }\ControlFlowTok{function}\NormalTok{(x) \{}
\NormalTok{  ux <-}\StringTok{ }\KeywordTok{unique}\NormalTok{(x)}
\NormalTok{  ux[}\KeywordTok{which.max}\NormalTok{(}\KeywordTok{tabulate}\NormalTok{(}\KeywordTok{match}\NormalTok{(x, ux)))]}
\NormalTok{\}}

\NormalTok{funkcja_pomiary_statystyczne <-}\StringTok{ }\ControlFlowTok{function}\NormalTok{(wektor)\{}
\NormalTok{  srednia_arytm <-}\StringTok{ }\KeywordTok{mean}\NormalTok{(wektor)}
\NormalTok{  mediana <-}\StringTok{ }\KeywordTok{median}\NormalTok{(wektor)}
\NormalTok{  dominanta <-}\StringTok{ }\KeywordTok{funkcja_dominanta}\NormalTok{(wektor)}
\NormalTok{  kwartyl1 <-}\StringTok{ }\KeywordTok{as.numeric}\NormalTok{(}\KeywordTok{quantile}\NormalTok{(wektor, }\DataTypeTok{probs =} \FloatTok{0.25}\NormalTok{))}
\NormalTok{  kwartyl3 <-}\StringTok{ }\KeywordTok{as.numeric}\NormalTok{(}\KeywordTok{quantile}\NormalTok{(wektor, }\DataTypeTok{probs =} \FloatTok{0.75}\NormalTok{))}
\NormalTok{  wariancja <-}\StringTok{ }\KeywordTok{var}\NormalTok{(wektor)}
\NormalTok{  wariancja_Obc <-}\StringTok{ }\NormalTok{wariancja }\OperatorTok{*}\StringTok{ }\NormalTok{(}\KeywordTok{length}\NormalTok{(wektor) }\OperatorTok{-}\StringTok{ }\DecValTok{1}\NormalTok{)}\OperatorTok{/}\KeywordTok{length}\NormalTok{(wektor)}
\NormalTok{  odchyl_stand <-}\StringTok{ }\KeywordTok{sqrt}\NormalTok{(wariancja)}
\NormalTok{  odchyl_stand_Obc <-}\StringTok{ }\KeywordTok{sqrt}\NormalTok{ (wariancja_Obc)}
\NormalTok{  odchyl_przec <-}\StringTok{ }\KeywordTok{sum}\NormalTok{(}\KeywordTok{abs}\NormalTok{(wektor }\OperatorTok{-}\StringTok{ }\NormalTok{srednia_arytm))}\OperatorTok{/}\KeywordTok{length}\NormalTok{(wektor)}
\NormalTok{  odchyl_cwiart <-}\StringTok{ }\NormalTok{(kwartyl3 }\OperatorTok{-}\StringTok{ }\NormalTok{kwartyl1)}\OperatorTok{/}\DecValTok{2}
\NormalTok{  klas_wspol_zmienn <-}\StringTok{ }\NormalTok{odchyl_stand}\OperatorTok{/}\NormalTok{srednia_arytm}
\NormalTok{  pozyc_wspol_zmienn <-}\StringTok{ }\NormalTok{odchyl_cwiart}\OperatorTok{/}\NormalTok{mediana}
\NormalTok{  kurtoza <-}\StringTok{ }\KeywordTok{kurtosis}\NormalTok{(wektor)}
\NormalTok{  eksces <-}\StringTok{ }\NormalTok{kurtoza }\OperatorTok{-}\StringTok{ }\DecValTok{3}
\NormalTok{  wsk_asymetrii <-}\StringTok{ }\NormalTok{kwartyl3 }\OperatorTok{-}\StringTok{ }\NormalTok{(}\DecValTok{2} \OperatorTok{*}\StringTok{ }\NormalTok{mediana) }\OperatorTok{+}\StringTok{ }\NormalTok{kwartyl1}
\NormalTok{  wspol_asymertii <-}\StringTok{ }\NormalTok{wsk_asymetrii }\OperatorTok{/}\StringTok{ }\NormalTok{(}\DecValTok{2} \OperatorTok{*}\StringTok{ }\NormalTok{odchyl_cwiart)}
\NormalTok{  dol_typ_obsz_zmienn <-}\StringTok{ }\NormalTok{srednia_arytm }\OperatorTok{-}\StringTok{ }\NormalTok{odchyl_stand}
\NormalTok{  gor_typ_obsz_zmienn <-}\StringTok{ }\NormalTok{srednia_arytm }\OperatorTok{+}\StringTok{ }\NormalTok{odchyl_stand}
  
\NormalTok{  rezultat <-}\StringTok{ }\KeywordTok{c}\NormalTok{(srednia_arytm, mediana, dominanta, kwartyl1, kwartyl3, wariancja,}
\NormalTok{                wariancja_Obc, odchyl_stand, odchyl_stand_Obc, odchyl_przec,}
\NormalTok{                odchyl_cwiart, klas_wspol_zmienn, pozyc_wspol_zmienn, kurtoza,}
\NormalTok{                eksces, wsk_asymetrii, wspol_asymertii, dol_typ_obsz_zmienn,}
\NormalTok{                gor_typ_obsz_zmienn)}
  
  \KeywordTok{return}\NormalTok{(rezultat)}
\NormalTok{\}}

\NormalTok{mpg_ <-}\StringTok{ }\KeywordTok{round}\NormalTok{( }\KeywordTok{funkcja_pomiary_statystyczne}\NormalTok{(Autko}\OperatorTok{$}\NormalTok{mpg), }\DataTypeTok{digits =} \DecValTok{4}\NormalTok{)}
\NormalTok{acceleration_ <-}\StringTok{ }\KeywordTok{round}\NormalTok{(}\KeywordTok{funkcja_pomiary_statystyczne}\NormalTok{(Autko}\OperatorTok{$}\NormalTok{acceleration), }\DataTypeTok{digits =} \DecValTok{4}\NormalTok{)}
\NormalTok{displacement_ <-}\StringTok{ }\KeywordTok{round}\NormalTok{(}\KeywordTok{funkcja_pomiary_statystyczne}\NormalTok{(Autko}\OperatorTok{$}\NormalTok{displacement), }\DataTypeTok{digits =} \DecValTok{4}\NormalTok{)}
\NormalTok{horsepower_ <-}\StringTok{  }\KeywordTok{round}\NormalTok{(}\KeywordTok{funkcja_pomiary_statystyczne}\NormalTok{(Autko}\OperatorTok{$}\NormalTok{horsepower), }\DataTypeTok{digits =} \DecValTok{4}\NormalTok{)}
\NormalTok{weight_ <-}\StringTok{ }\KeywordTok{round}\NormalTok{(}\KeywordTok{funkcja_pomiary_statystyczne}\NormalTok{(Autko}\OperatorTok{$}\NormalTok{weight), }\DataTypeTok{digits =} \DecValTok{4}\NormalTok{)}

\NormalTok{tmp <-}\StringTok{ }\KeywordTok{data.frame}\NormalTok{(mpg_, acceleration_,displacement_, horsepower_, weight_)}
\KeywordTok{row.names}\NormalTok{(tmp) <-}\StringTok{ }\KeywordTok{c}\NormalTok{(}\StringTok{"srednia arytmetyczna: "}\NormalTok{, }\StringTok{"mediana: "}\NormalTok{, }\StringTok{"dominanta: "}\NormalTok{, }\StringTok{"kwartyl 0.25: "}\NormalTok{, }\StringTok{"kwartyl 0.75: "}\NormalTok{,}
                    \StringTok{"wariancja: "}\NormalTok{, }\StringTok{"wariancja obciazona: "}\NormalTok{, }\StringTok{"odchylenie standardowe: "}\NormalTok{,}
                    \StringTok{"odchylenie standardowe obciazone: "}\NormalTok{, }\StringTok{"odchylenie przecietne: "}\NormalTok{,}
                    \StringTok{"odchylenie cwiartkowe: "}\NormalTok{, }\StringTok{"klasyczny wspolczynnik zmiennosci: "}\NormalTok{,}
                    \StringTok{"pozycyjny wspolczynnik zmiennosci: "}\NormalTok{, }\StringTok{"kurtoza: "}\NormalTok{, }\StringTok{"eksces: "}\NormalTok{,}
                    \StringTok{"wskaznik asymetrii: "}\NormalTok{, }\StringTok{"wspolczynnik asymetrii: "}\NormalTok{,}
                    \StringTok{"dolna granica typowego obszaru zmiennosci: "}\NormalTok{,}
                    \StringTok{"gorna granica typowego obszaru zmiennosci: "}\NormalTok{)}

\KeywordTok{print}\NormalTok{(tmp)}
\end{Highlighting}
\end{Shaded}

\begin{verbatim}
##                                                mpg_ acceleration_ displacement_
## srednia arytmetyczna:                       23.5146       15.5681      193.4259
## mediana:                                    23.0000       15.5000      148.5000
## dominanta:                                  13.0000       14.5000       97.0000
## kwartyl 0.25:                               17.5000       13.8250      104.2500
## kwartyl 0.75:                               29.0000       17.1750      262.0000
## wariancja:                                  61.0896        7.6048    10872.1992
## wariancja obciazona:                        60.9361        7.5857    10844.8821
## odchylenie standardowe:                      7.8160        2.7577      104.2698
## odchylenie standardowe obciazone:            7.8062        2.7542      104.1388
## odchylenie przecietne:                       6.5334        2.1426       91.3526
## odchylenie cwiartkowe:                       5.7500        1.6750       78.8750
## klasyczny wspolczynnik zmiennosci:           0.3324        0.1771        0.5391
## pozycyjny wspolczynnik zmiennosci:           0.2500        0.1081        0.5311
## kurtoza:                                    -0.5319        0.3821       -0.7636
## eksces:                                     -3.5319       -2.6179       -3.7636
## wskaznik asymetrii:                          0.5000        0.0000       69.2500
## wspolczynnik asymetrii:                      0.0435        0.0000        0.4390
## dolna granica typowego obszaru zmiennosci:  15.6986       12.8104       89.1560
## gorna granica typowego obszaru zmiennosci:  31.3306       18.3258      297.6957
##                                             horsepower_     weight_
## srednia arytmetyczna:                          104.4694   2970.4246
## mediana:                                        95.0000   2803.5000
## dominanta:                                     150.0000   2130.0000
## kwartyl 0.25:                                   76.0000   2223.7500
## kwartyl 0.75:                                  125.0000   3608.0000
## wariancja:                                    1459.1779 717140.9905
## wariancja obciazona:                          1455.5116 715339.1287
## odchylenie standardowe:                         38.1992    846.8418
## odchylenie standardowe obciazone:               38.1512    845.7772
## odchylenie przecietne:                          29.9022    717.9216
## odchylenie cwiartkowe:                          24.5000    692.1250
## klasyczny wspolczynnik zmiennosci:               0.3656      0.2851
## pozycyjny wspolczynnik zmiennosci:               0.2579      0.2469
## kurtoza:                                         0.7103     -0.8018
## eksces:                                         -2.2897     -3.8018
## wskaznik asymetrii:                             11.0000    224.7500
## wspolczynnik asymetrii:                          0.2245      0.1624
## dolna granica typowego obszaru zmiennosci:      66.2702   2123.5828
## gorna granica typowego obszaru zmiennosci:     142.6686   3817.2664
\end{verbatim}

\hypertarget{histogramy}{%
\subsection{Histogramy}\label{histogramy}}

\begin{Shaded}
\begin{Highlighting}[]
\NormalTok{wynik <-}\StringTok{ }\ControlFlowTok{function}\NormalTok{(x)}
\NormalTok{\{}
\NormalTok{  min <-}\StringTok{ }\KeywordTok{min}\NormalTok{(x, }\DataTypeTok{na.rm =} \OtherTok{TRUE}\NormalTok{)}
\NormalTok{  max <-}\StringTok{ }\KeywordTok{max}\NormalTok{(x, }\DataTypeTok{na.rm =} \OtherTok{TRUE}\NormalTok{)}
\NormalTok{  zakres <-}\StringTok{ }\KeywordTok{as.numeric}\NormalTok{(max }\OperatorTok{-}\StringTok{ }\NormalTok{min)}
\NormalTok{  ilosc <-}\StringTok{ }\KeywordTok{as.numeric}\NormalTok{(}\KeywordTok{nrow}\NormalTok{(Autko))}
\NormalTok{  pierwiastek <-}\StringTok{ }\KeywordTok{sqrt}\NormalTok{(ilosc)}
\NormalTok{  pierwiastek <-}\StringTok{ }\KeywordTok{ceiling}\NormalTok{(pierwiastek)}
\NormalTok{  szer <-}\StringTok{ }\NormalTok{zakres }\OperatorTok{/}\StringTok{ }\NormalTok{pierwiastek}
\NormalTok{  szer <-szer[}\DecValTok{1}\NormalTok{]}
\NormalTok{  pkt =}\StringTok{ }\KeywordTok{seq}\NormalTok{(min, max, }\DataTypeTok{by =}\NormalTok{ szer)}
\NormalTok{  rezultat <-}\StringTok{ }\KeywordTok{c}\NormalTok{(min, max, szer, pkt)}
  \KeywordTok{return}\NormalTok{(rezultat)}
\NormalTok{\}}
\CommentTok{#dane zapisane w wektorze}
\NormalTok{mpg<-}\KeywordTok{wynik}\NormalTok{(Autko}\OperatorTok{$}\NormalTok{mpg)}
\NormalTok{dis<-}\KeywordTok{wynik}\NormalTok{(Autko}\OperatorTok{$}\NormalTok{displacement)}
\NormalTok{kg<-}\KeywordTok{wynik}\NormalTok{(Autko}\OperatorTok{$}\NormalTok{weight)}
\NormalTok{pow<-}\KeywordTok{wynik}\NormalTok{(Autko}\OperatorTok{$}\NormalTok{horsepower)}
\NormalTok{acc<-}\KeywordTok{wynik}\NormalTok{(Autko}\OperatorTok{$}\NormalTok{acceleration)}
\CommentTok{#przedziały}
\NormalTok{przedzialmpg <-}\StringTok{ }\KeywordTok{cut}\NormalTok{(Autko}\OperatorTok{$}\NormalTok{mpg, mpg[}\OperatorTok{-}\KeywordTok{c}\NormalTok{(}\DecValTok{1}\NormalTok{,}\DecValTok{2}\NormalTok{,}\DecValTok{3}\NormalTok{)], }\DataTypeTok{right =} \OtherTok{FALSE}\NormalTok{, }\DataTypeTok{include.lowest =} \OtherTok{TRUE}\NormalTok{)}
\NormalTok{przedzialdis <-}\StringTok{ }\KeywordTok{cut}\NormalTok{(Autko}\OperatorTok{$}\NormalTok{displacement, dis[}\OperatorTok{-}\KeywordTok{c}\NormalTok{(}\DecValTok{1}\NormalTok{,}\DecValTok{2}\NormalTok{,}\DecValTok{3}\NormalTok{)], }\DataTypeTok{right =} \OtherTok{FALSE}\NormalTok{, }\DataTypeTok{include.lowest =} \OtherTok{TRUE}\NormalTok{)}
\NormalTok{przedzialpow <-}\StringTok{ }\KeywordTok{cut}\NormalTok{(Autko}\OperatorTok{$}\NormalTok{horsepower, pow[}\OperatorTok{-}\KeywordTok{c}\NormalTok{(}\DecValTok{1}\NormalTok{,}\DecValTok{2}\NormalTok{,}\DecValTok{3}\NormalTok{)], }\DataTypeTok{right =} \OtherTok{FALSE}\NormalTok{, }\DataTypeTok{include.lowest =} \OtherTok{TRUE}\NormalTok{)}
\NormalTok{przedzialkg <-}\StringTok{ }\KeywordTok{cut}\NormalTok{(Autko}\OperatorTok{$}\NormalTok{weight, kg[}\OperatorTok{-}\KeywordTok{c}\NormalTok{(}\DecValTok{1}\NormalTok{,}\DecValTok{2}\NormalTok{,}\DecValTok{3}\NormalTok{)], }\DataTypeTok{right =} \OtherTok{FALSE}\NormalTok{, }\DataTypeTok{include.lowest =} \OtherTok{TRUE}\NormalTok{)}
\NormalTok{przedzialacc <-}\StringTok{ }\KeywordTok{cut}\NormalTok{(Autko}\OperatorTok{$}\NormalTok{acceleration, acc[}\OperatorTok{-}\KeywordTok{c}\NormalTok{(}\DecValTok{1}\NormalTok{,}\DecValTok{2}\NormalTok{,}\DecValTok{3}\NormalTok{)], }\DataTypeTok{right =} \OtherTok{FALSE}\NormalTok{, }\DataTypeTok{include.lowest =} \OtherTok{TRUE}\NormalTok{)}
\CommentTok{# szeregi rozdzielcze}
\NormalTok{szeregmpg <-}\StringTok{ }\KeywordTok{table}\NormalTok{(przedzialmpg)}
\NormalTok{szeregdis <-}\StringTok{ }\KeywordTok{table}\NormalTok{(przedzialdis)}
\NormalTok{szeregpow <-}\StringTok{ }\KeywordTok{table}\NormalTok{(przedzialpow)}
\NormalTok{szeregkg <-}\StringTok{ }\KeywordTok{table}\NormalTok{(przedzialkg)}
\NormalTok{szeregacc <-}\StringTok{ }\KeywordTok{table}\NormalTok{(przedzialacc)}
\CommentTok{#histogramy}
\KeywordTok{ggplot}\NormalTok{(Autko, }\KeywordTok{aes}\NormalTok{(}\DataTypeTok{x=}\NormalTok{mpg)) }\OperatorTok{+}\StringTok{ }\KeywordTok{geom_histogram}\NormalTok{(}\DataTypeTok{breaks =}\NormalTok{  mpg[}\OperatorTok{-}\KeywordTok{c}\NormalTok{(}\DecValTok{1}\NormalTok{,}\DecValTok{2}\NormalTok{,}\DecValTok{3}\NormalTok{)] , }\KeywordTok{aes}\NormalTok{(}\DataTypeTok{y=}\NormalTok{..density..),  }\DataTypeTok{colour=}\StringTok{"black"}\NormalTok{, }\DataTypeTok{fill=}\StringTok{"white"}\NormalTok{) }\OperatorTok{+}\StringTok{ }\KeywordTok{geom_density}\NormalTok{(}\DataTypeTok{alpha=}\NormalTok{.}\DecValTok{2}\NormalTok{, }\DataTypeTok{fill=}\StringTok{"blue"}\NormalTok{) }\OperatorTok{+}\StringTok{ }\KeywordTok{labs}\NormalTok{(}\DataTypeTok{title =} \StringTok{"Histogram mpg"}\NormalTok{)}
\end{Highlighting}
\end{Shaded}

\includegraphics{ProjektMS_files/figure-latex/unnamed-chunk-4-1.pdf}

\begin{Shaded}
\begin{Highlighting}[]
\KeywordTok{ggplot}\NormalTok{(Autko, }\KeywordTok{aes}\NormalTok{(}\DataTypeTok{x=}\NormalTok{displacement)) }\OperatorTok{+}\StringTok{ }\KeywordTok{geom_histogram}\NormalTok{(}\DataTypeTok{breaks =}\NormalTok{ dis[}\OperatorTok{-}\KeywordTok{c}\NormalTok{(}\DecValTok{1}\NormalTok{,}\DecValTok{2}\NormalTok{,}\DecValTok{3}\NormalTok{)], }\KeywordTok{aes}\NormalTok{(}\DataTypeTok{y=}\NormalTok{..density..),  }\DataTypeTok{colour=}\StringTok{"black"}\NormalTok{, }\DataTypeTok{fill=}\StringTok{"white"}\NormalTok{) }\OperatorTok{+}\StringTok{ }\KeywordTok{geom_density}\NormalTok{(}\DataTypeTok{alpha=}\NormalTok{.}\DecValTok{2}\NormalTok{, }\DataTypeTok{fill=}\StringTok{"blue"}\NormalTok{) }\OperatorTok{+}\StringTok{ }\KeywordTok{labs}\NormalTok{(}\DataTypeTok{title =} \StringTok{"Histogram displacement"}\NormalTok{)}
\end{Highlighting}
\end{Shaded}

\includegraphics{ProjektMS_files/figure-latex/unnamed-chunk-4-2.pdf}

\begin{Shaded}
\begin{Highlighting}[]
\KeywordTok{ggplot}\NormalTok{(Autko, }\KeywordTok{aes}\NormalTok{(}\DataTypeTok{x=}\NormalTok{horsepower)) }\OperatorTok{+}\StringTok{  }\KeywordTok{geom_histogram}\NormalTok{(}\DataTypeTok{breaks =}\NormalTok{ pow[}\OperatorTok{-}\KeywordTok{c}\NormalTok{(}\DecValTok{1}\NormalTok{,}\DecValTok{2}\NormalTok{,}\DecValTok{3}\NormalTok{)], }\KeywordTok{aes}\NormalTok{(}\DataTypeTok{y=}\NormalTok{..density..),  }\DataTypeTok{colour=}\StringTok{"black"}\NormalTok{, }\DataTypeTok{fill=}\StringTok{"white"}\NormalTok{) }\OperatorTok{+}\StringTok{ }\KeywordTok{geom_density}\NormalTok{(}\DataTypeTok{alpha=}\NormalTok{.}\DecValTok{2}\NormalTok{, }\DataTypeTok{fill=}\StringTok{"blue"}\NormalTok{) }\OperatorTok{+}\StringTok{ }\KeywordTok{labs}\NormalTok{(}\DataTypeTok{title =} \StringTok{"Histogram horsepower"}\NormalTok{)}
\end{Highlighting}
\end{Shaded}

\includegraphics{ProjektMS_files/figure-latex/unnamed-chunk-4-3.pdf}

\begin{Shaded}
\begin{Highlighting}[]
\KeywordTok{ggplot}\NormalTok{(Autko, }\KeywordTok{aes}\NormalTok{(}\DataTypeTok{x=}\NormalTok{weight)) }\OperatorTok{+}\StringTok{ }\KeywordTok{geom_histogram}\NormalTok{(}\DataTypeTok{breaks =}\NormalTok{ kg[}\OperatorTok{-}\KeywordTok{c}\NormalTok{(}\DecValTok{1}\NormalTok{,}\DecValTok{2}\NormalTok{,}\DecValTok{3}\NormalTok{)], }\KeywordTok{aes}\NormalTok{(}\DataTypeTok{y=}\NormalTok{..density..),  }\DataTypeTok{colour=}\StringTok{"black"}\NormalTok{, }\DataTypeTok{fill=}\StringTok{"white"}\NormalTok{) }\OperatorTok{+}\StringTok{ }\KeywordTok{geom_density}\NormalTok{(}\DataTypeTok{alpha=}\NormalTok{.}\DecValTok{2}\NormalTok{, }\DataTypeTok{fill=}\StringTok{"blue"}\NormalTok{) }\OperatorTok{+}\StringTok{ }\KeywordTok{labs}\NormalTok{(}\DataTypeTok{title =} \StringTok{"Histogram weight"}\NormalTok{)}
\end{Highlighting}
\end{Shaded}

\includegraphics{ProjektMS_files/figure-latex/unnamed-chunk-4-4.pdf}

\begin{Shaded}
\begin{Highlighting}[]
\KeywordTok{ggplot}\NormalTok{(Autko, }\KeywordTok{aes}\NormalTok{(}\DataTypeTok{x=}\NormalTok{acceleration)) }\OperatorTok{+}\StringTok{ }\KeywordTok{geom_histogram}\NormalTok{(}\DataTypeTok{breaks =}\NormalTok{acc[}\OperatorTok{-}\KeywordTok{c}\NormalTok{(}\DecValTok{1}\NormalTok{,}\DecValTok{2}\NormalTok{,}\DecValTok{3}\NormalTok{)]   , }\KeywordTok{aes}\NormalTok{(}\DataTypeTok{y=}\NormalTok{..density..),  }\DataTypeTok{colour=}\StringTok{"black"}\NormalTok{, }\DataTypeTok{fill=}\StringTok{"white"}\NormalTok{) }\OperatorTok{+}\StringTok{ }\KeywordTok{geom_density}\NormalTok{(}\DataTypeTok{alpha=}\NormalTok{.}\DecValTok{2}\NormalTok{, }\DataTypeTok{fill=}\StringTok{"blue"}\NormalTok{) }\OperatorTok{+}\StringTok{ }\KeywordTok{labs}\NormalTok{(}\DataTypeTok{title =} \StringTok{"Histogram acceleration"}\NormalTok{)}
\end{Highlighting}
\end{Shaded}

\includegraphics{ProjektMS_files/figure-latex/unnamed-chunk-4-5.pdf}

\hypertarget{cumulative-distribution-function-plot}{%
\subsection{Cumulative distribution function
plot}\label{cumulative-distribution-function-plot}}

\hypertarget{wykres-funkcji-rozkux142adu-skumulowanego}{%
\subsection{Wykres funkcji rozkładu
skumulowanego}\label{wykres-funkcji-rozkux142adu-skumulowanego}}

Procent skumulowany - to statystyczna miara, określająca jaki odsetek
``osób'', w tym przypadku samochodów, uzyskał pewien zakres wyników. Jak
sama nazwa wskazuje jest to procent złożony z dodawania procentów dla
pojedyńczych kategorii - następuje kumulacja.

Dzięki temu w łatwy sposób (bez dodawania) określić
odsetek(procent/prawdopodobieństwo) samochodów przyjmujących pewien
zakres, licząc od początku do danej wartości.

Aby uzyskać taki wykres należy:

\begin{itemize}
\item
  uzyskać dane i obliczyć kluczowe statystyki podsomowujące,
\item
  wyodrębnić wektor danych ``mpg'' dla Auto-Mpg,
\end{itemize}

\begin{Shaded}
\begin{Highlighting}[]
\NormalTok{mpg =}\StringTok{ }\NormalTok{Autko}\OperatorTok{$}\NormalTok{mpg}
\end{Highlighting}
\end{Shaded}

\begin{itemize}
\tightlist
\item
  obliczyć liczbę nie brakujących wartości w ``mpg''
\end{itemize}

\begin{Shaded}
\begin{Highlighting}[]
\NormalTok{n =}\StringTok{ }\KeywordTok{sum}\NormalTok{(}\OperatorTok{!}\KeywordTok{is.na}\NormalTok{(mpg));}
\end{Highlighting}
\end{Shaded}

\begin{itemize}
\tightlist
\item
  uzyskać empiryczne wartości CDF
\end{itemize}

\begin{Shaded}
\begin{Highlighting}[]
\NormalTok{mpg.ecdf =}\StringTok{ }\KeywordTok{ecdf}\NormalTok{(mpg)}
\end{Highlighting}
\end{Shaded}

Teraz możemy wykreślić empiryczną funkcję rozkłądu skumulowanego (za
pomocą ecdf() i plot()):

\begin{Shaded}
\begin{Highlighting}[]
\KeywordTok{plot}\NormalTok{(mpg.ecdf, }\DataTypeTok{xlab =} \StringTok{'MPG - Miles Per Gallon'}\NormalTok{, }\DataTypeTok{ylab =} \StringTok{'Prawdopodobienstwo'}\NormalTok{, }\DataTypeTok{main =} \StringTok{'Empiryczny rozkład skumulowany}\CharTok{\textbackslash{}n}\StringTok{MPG samochodów')}
\end{Highlighting}
\end{Shaded}

\includegraphics{ProjektMS_files/figure-latex/unnamed-chunk-10-1.pdf}

Zatem samochodóW spalających 20 galonów na milę i mniej jest około 38\%.

Samochodów spalających 30 galonów na milę jest 80\%.

Dla ostatniej możliwej kategorii/opcji procent zawsze będzie wynosił
100\%, gdyż ``wyczerpuje'' on wszystkie pozostałe kategorie/opcje.

\hypertarget{qq-plots}{%
\subsection{QQ plots}\label{qq-plots}}

\hypertarget{wykresy-qq}{%
\subsection{Wykresy QQ}\label{wykresy-qq}}

Wykres QQ (kwantylowo-kwantylowy) ukazuje nam korelację pomiędzy daną
próbką, a rozkładem normalnym.Rysowana jest również 45 stopniowa linia
odniesienia. Wykresy QQ to narzędzie graficzne pomagające nam ocenić,
czy zbiór danych pochodzi z jakiegoś teoretycznego rozkładu (normalny,
wykładniczy). Jest to kontrola wizualna, a nie hermetyczny dowód, więc
jest ona subiektywna.

Aby uzyskać taki wykres należy:

\begin{itemize}
\item
  utworzyć normalny wykres zmiennej -\textgreater{} qqnorm(),
\item
  dodać linię odniesiena -\textgreater{} qqline(),
\end{itemize}

Teraz możemy utworzyć wykres QQ:

\hypertarget{dla-mpg}{%
\subparagraph{- dla MPG,}\label{dla-mpg}}

\begin{Shaded}
\begin{Highlighting}[]
\KeywordTok{qqnorm}\NormalTok{(Autko}\OperatorTok{$}\NormalTok{mpg, }\DataTypeTok{pch=}\DecValTok{1}\NormalTok{, }\DataTypeTok{frame=}\OtherTok{FALSE}\NormalTok{, }\DataTypeTok{main=}\StringTok{"QQ plot - MPG"}\NormalTok{)}
\KeywordTok{qqline}\NormalTok{(Autko}\OperatorTok{$}\NormalTok{mpg, }\DataTypeTok{col=}\StringTok{'red'}\NormalTok{, }\DataTypeTok{lwd=}\DecValTok{2}\NormalTok{)}
\end{Highlighting}
\end{Shaded}

\includegraphics{ProjektMS_files/figure-latex/unnamed-chunk-17-1.pdf}

\hypertarget{dla-displacement}{%
\subparagraph{- dla Displacement,}\label{dla-displacement}}

\begin{Shaded}
\begin{Highlighting}[]
\KeywordTok{qqnorm}\NormalTok{(Autko}\OperatorTok{$}\NormalTok{displacement, }\DataTypeTok{pch=}\DecValTok{1}\NormalTok{, }\DataTypeTok{frame=}\OtherTok{FALSE}\NormalTok{, }\DataTypeTok{main=}\StringTok{"QQ plot - Displacement"}\NormalTok{)}
\KeywordTok{qqline}\NormalTok{(Autko}\OperatorTok{$}\NormalTok{displacement, }\DataTypeTok{col=}\StringTok{'red'}\NormalTok{, }\DataTypeTok{lwd=}\DecValTok{2}\NormalTok{)}
\end{Highlighting}
\end{Shaded}

\includegraphics{ProjektMS_files/figure-latex/unnamed-chunk-19-1.pdf}

\hypertarget{dla-horsepower}{%
\subparagraph{- dla Horsepower,}\label{dla-horsepower}}

\begin{Shaded}
\begin{Highlighting}[]
\KeywordTok{qqnorm}\NormalTok{(Autko}\OperatorTok{$}\NormalTok{horsepower, }\DataTypeTok{pch=}\DecValTok{1}\NormalTok{, }\DataTypeTok{frame=}\OtherTok{FALSE}\NormalTok{, }\DataTypeTok{main=}\StringTok{"QQ plot - Horsepower"}\NormalTok{)}
\KeywordTok{qqline}\NormalTok{(Autko}\OperatorTok{$}\NormalTok{horsepower, }\DataTypeTok{col=}\StringTok{'red'}\NormalTok{, }\DataTypeTok{lwd=}\DecValTok{2}\NormalTok{)}
\end{Highlighting}
\end{Shaded}

\includegraphics{ProjektMS_files/figure-latex/unnamed-chunk-21-1.pdf}

\hypertarget{dla-weight}{%
\subparagraph{- dla Weight,}\label{dla-weight}}

\begin{Shaded}
\begin{Highlighting}[]
\KeywordTok{qqnorm}\NormalTok{(Autko}\OperatorTok{$}\NormalTok{weight, }\DataTypeTok{pch=}\DecValTok{1}\NormalTok{, }\DataTypeTok{frame=}\OtherTok{FALSE}\NormalTok{, }\DataTypeTok{main=}\StringTok{"QQ plot - Weight"}\NormalTok{)}
\KeywordTok{qqline}\NormalTok{(Autko}\OperatorTok{$}\NormalTok{weight, }\DataTypeTok{col=}\StringTok{'red'}\NormalTok{, }\DataTypeTok{lwd=}\DecValTok{2}\NormalTok{)}
\end{Highlighting}
\end{Shaded}

\includegraphics{ProjektMS_files/figure-latex/unnamed-chunk-23-1.pdf}

\hypertarget{dla-acceleration.}{%
\subparagraph{- dla Acceleration.}\label{dla-acceleration.}}

\begin{Shaded}
\begin{Highlighting}[]
\KeywordTok{qqnorm}\NormalTok{(Autko}\OperatorTok{$}\NormalTok{acceleration, }\DataTypeTok{pch=}\DecValTok{1}\NormalTok{, }\DataTypeTok{frame=}\OtherTok{FALSE}\NormalTok{, }\DataTypeTok{main=}\StringTok{"QQ plot - Acceleration"}\NormalTok{)}
\KeywordTok{qqline}\NormalTok{(Autko}\OperatorTok{$}\NormalTok{acceleration, }\DataTypeTok{col=}\StringTok{'red'}\NormalTok{, }\DataTypeTok{lwd=}\DecValTok{2}\NormalTok{)}
\end{Highlighting}
\end{Shaded}

\includegraphics{ProjektMS_files/figure-latex/unnamed-chunk-25-1.pdf}

Jeśli wszystkie punkty opadają w przybliżeniu wzdłuż linii odniesienia
to możemy założyć normalność. Jesli natomiast punkty tworzą krzywą
zamiast linię prostą to w tym momencie mamy do czynienia z wypaczeniem
danych próbki.

\hypertarget{scatterplot-matrix-by-class}{%
\subsection{Scatterplot matrix (by
class)}\label{scatterplot-matrix-by-class}}

\hypertarget{wykres-macierzy-rozrzutu-wedux142ug-klasy}{%
\subsection{Wykres macierzy rozrzutu (według
klasy)}\label{wykres-macierzy-rozrzutu-wedux142ug-klasy}}

Macierz rozrzutu umożliwia nam zwizualizowanie korelacji małych zestawów
danych. Wykres macierzy rozrzutu pokazuje nam wszystkie pary wykresów
rozrzutu zmiennych w jednym widoku w formacie macierzy.

Aby uzyskać wykres macierzy rozrzutu należy:

\begin{itemize}
\tightlist
\item
  utworzyć podstawowy wykres za pomocą -\textgreater{} pairs(),
\end{itemize}

\begin{Shaded}
\begin{Highlighting}[]
\KeywordTok{pairs}\NormalTok{(Autko[}\KeywordTok{c}\NormalTok{(}\DecValTok{1}\NormalTok{,}\DecValTok{3}\NormalTok{,}\DecValTok{4}\NormalTok{,}\DecValTok{5}\NormalTok{,}\DecValTok{6}\NormalTok{)], }\DataTypeTok{pch =} \DecValTok{19}\NormalTok{)}
\end{Highlighting}
\end{Shaded}

\includegraphics{ProjektMS_files/figure-latex/unnamed-chunk-28-1.pdf}

\begin{itemize}
\tightlist
\item
  można usunąć dolną część wykresu,
\end{itemize}

\begin{Shaded}
\begin{Highlighting}[]
\KeywordTok{pairs}\NormalTok{(Autko[}\KeywordTok{c}\NormalTok{(}\DecValTok{1}\NormalTok{,}\DecValTok{3}\NormalTok{,}\DecValTok{4}\NormalTok{,}\DecValTok{5}\NormalTok{,}\DecValTok{6}\NormalTok{)], }\DataTypeTok{pch =} \DecValTok{19}\NormalTok{, }\DataTypeTok{lower.panel=}\OtherTok{NULL}\NormalTok{)}
\end{Highlighting}
\end{Shaded}

\includegraphics{ProjektMS_files/figure-latex/unnamed-chunk-30-1.pdf}

\begin{itemize}
\tightlist
\item
  można pokolorować punkty poprzez poszczególną klasę (np. origin) w
  celu lepszego zobrazowania korelacji,
\end{itemize}

\begin{Shaded}
\begin{Highlighting}[]
\NormalTok{my_cols <-}\StringTok{ }\KeywordTok{c}\NormalTok{(}\StringTok{"#FF0000"}\NormalTok{, }\StringTok{"#00FF00"}\NormalTok{, }\StringTok{"#0000FF"}\NormalTok{)  }
\KeywordTok{pairs}\NormalTok{(Autko[}\KeywordTok{c}\NormalTok{(}\DecValTok{1}\NormalTok{,}\DecValTok{3}\NormalTok{,}\DecValTok{4}\NormalTok{,}\DecValTok{5}\NormalTok{,}\DecValTok{6}\NormalTok{)], }\DataTypeTok{pch =} \DecValTok{19}\NormalTok{,  }\DataTypeTok{cex =} \FloatTok{0.5}\NormalTok{,}
      \DataTypeTok{col =}\NormalTok{ my_cols[Autko}\OperatorTok{$}\NormalTok{origin],}
      \DataTypeTok{lower.panel=}\OtherTok{NULL}\NormalTok{)}
\end{Highlighting}
\end{Shaded}

\includegraphics{ProjektMS_files/figure-latex/unnamed-chunk-32-1.pdf}

\hypertarget{regresja-liniowa}{%
\subsection{Regresja Liniowa}\label{regresja-liniowa}}

Rregresja Liniowa podobnie jak klasyfikacja należy do zagadnień uczenia
z nadzorem (supervised learning). Uczymy model tzn. na danych
treningowych, ze znanymi wartościami Y. Model ma możliwie najtrafniej
przewidywać wartości Y na nowych danych

Losowanie ziarna i wybieranie nowego testu i trainingu

\begin{Shaded}
\begin{Highlighting}[]
\KeywordTok{set.seed}\NormalTok{(}\DecValTok{100}\NormalTok{)}
\NormalTok{indexes <-}\StringTok{ }\KeywordTok{sample}\NormalTok{(}\KeywordTok{nrow}\NormalTok{(Autko), (}\FloatTok{0.7}\OperatorTok{*}\KeywordTok{nrow}\NormalTok{(Autko)), }\DataTypeTok{replace =} \OtherTok{FALSE}\NormalTok{)}
\NormalTok{trainData <-}\StringTok{ }\NormalTok{Autko[indexes, ]}
\NormalTok{testData <-}\StringTok{ }\NormalTok{Autko[}\OperatorTok{-}\NormalTok{indexes, ]}
\end{Highlighting}
\end{Shaded}

\begin{itemize}
\tightlist
\item
  Wykresy dla regresji
\end{itemize}

\begin{Shaded}
\begin{Highlighting}[]
\CommentTok{#regresja liniowa mpg i weight}
\KeywordTok{ggplot}\NormalTok{(}\DataTypeTok{data=}\NormalTok{ Autko,}\KeywordTok{aes}\NormalTok{(weight,mpg)) }\OperatorTok{+}\StringTok{ }\KeywordTok{geom_point}\NormalTok{()}\OperatorTok{+}\StringTok{ }\KeywordTok{geom_smooth}\NormalTok{(}\DataTypeTok{method=}\NormalTok{lm) }
\end{Highlighting}
\end{Shaded}

\begin{verbatim}
## `geom_smooth()` using formula 'y ~ x'
\end{verbatim}

\includegraphics{ProjektMS_files/figure-latex/unnamed-chunk-34-1.pdf}

\begin{Shaded}
\begin{Highlighting}[]
\CommentTok{#regresja liniowa  mpg i displacement}
\KeywordTok{ggplot}\NormalTok{(Autko,}\KeywordTok{aes}\NormalTok{(displacement,mpg)) }\OperatorTok{+}\KeywordTok{geom_point}\NormalTok{()}\OperatorTok{+}\KeywordTok{geom_smooth}\NormalTok{(}\DataTypeTok{method=}\NormalTok{lm) }
\end{Highlighting}
\end{Shaded}

\begin{verbatim}
## `geom_smooth()` using formula 'y ~ x'
\end{verbatim}

\includegraphics{ProjektMS_files/figure-latex/unnamed-chunk-34-2.pdf}

\begin{Shaded}
\begin{Highlighting}[]
\CommentTok{#regresja liniowa  weight i horsepower}
\KeywordTok{ggplot}\NormalTok{(Autko,}\KeywordTok{aes}\NormalTok{(weight, horsepower)) }\OperatorTok{+}\StringTok{ }\KeywordTok{geom_point}\NormalTok{() }\OperatorTok{+}\KeywordTok{geom_smooth}\NormalTok{(}\DataTypeTok{method =}\NormalTok{ lm)}
\end{Highlighting}
\end{Shaded}

\begin{verbatim}
## `geom_smooth()` using formula 'y ~ x'
\end{verbatim}

\includegraphics{ProjektMS_files/figure-latex/unnamed-chunk-34-3.pdf}

Budowanie nowych danych do stworzenia korelacji, wybranie tylko kolumn
numerycznych

\begin{Shaded}
\begin{Highlighting}[]
\NormalTok{newdata <-}\StringTok{ }\KeywordTok{cor}\NormalTok{(Autko[ , }\KeywordTok{c}\NormalTok{(}\StringTok{'mpg'}\NormalTok{,}\StringTok{'weight'}\NormalTok{, }\StringTok{'displacement'}\NormalTok{, }\StringTok{'horsepower'}\NormalTok{, }\StringTok{'acceleration'}\NormalTok{)], }\DataTypeTok{use=}\StringTok{'complete'}\NormalTok{)}
\KeywordTok{corrplot}\NormalTok{(newdata, }\DataTypeTok{method =} \StringTok{"number"}\NormalTok{)}
\end{Highlighting}
\end{Shaded}

\includegraphics{ProjektMS_files/figure-latex/unnamed-chunk-36-1.pdf}

Wykres korelacji

\begin{Shaded}
\begin{Highlighting}[]
\KeywordTok{ggcorrplot}\NormalTok{(newdata, }\DataTypeTok{hc.order =} \OtherTok{TRUE}\NormalTok{, }
           \DataTypeTok{type =} \StringTok{"upper"}\NormalTok{, }
           \DataTypeTok{lab =} \OtherTok{TRUE}\NormalTok{, }
           \DataTypeTok{lab_size =} \DecValTok{3}\NormalTok{, }
           \DataTypeTok{method=}\StringTok{"circle"}\NormalTok{, }
           \DataTypeTok{colors =} \KeywordTok{c}\NormalTok{(}\StringTok{"tomato2"}\NormalTok{, }\StringTok{"white"}\NormalTok{, }\StringTok{"springgreen3"}\NormalTok{), }
           \DataTypeTok{title=}\StringTok{"Korelacja"}\NormalTok{, }
           \DataTypeTok{ggtheme=}\NormalTok{theme_bw)}
\end{Highlighting}
\end{Shaded}

\includegraphics{ProjektMS_files/figure-latex/unnamed-chunk-38-1.pdf}
Mpg z wszystkimi innymi kolumnami koreluje na minusie, posiada też
wysoką korelację. Największa korelację ma displacement i weight równą
0.93

Tworzenie regresji liniowej dla całego zbioru z wieloma predykatorami

\begin{Shaded}
\begin{Highlighting}[]
\NormalTok{model <-}\StringTok{ }\KeywordTok{lm}\NormalTok{(mpg}\OperatorTok{~}\NormalTok{weight}\OperatorTok{+}\NormalTok{horsepower}\OperatorTok{+}\NormalTok{origin}\OperatorTok{+}\NormalTok{model_year}\OperatorTok{+}\NormalTok{displacement}\OperatorTok{+}\NormalTok{acceleration,}\DataTypeTok{data =}\NormalTok{ Autko)}
\KeywordTok{summary}\NormalTok{(model)}
\end{Highlighting}
\end{Shaded}

\begin{verbatim}
## 
## Call:
## lm(formula = mpg ~ weight + horsepower + origin + model_year + 
##     displacement + acceleration, data = Autko)
## 
## Residuals:
##    Min     1Q Median     3Q    Max 
## -9.213 -1.888 -0.066  1.863 11.683 
## 
## Coefficients:
##                Estimate Std. Error t value Pr(>|t|)    
## (Intercept)  36.4141653  1.9510793  18.664  < 2e-16 ***
## weight       -0.0063635  0.0006284 -10.127  < 2e-16 ***
## horsepower   -0.0214674  0.0127958  -1.678 0.094231 .  
## origin2       2.6899477  0.5229265   5.144 4.32e-07 ***
## origin3       2.5356695  0.5104774   4.967 1.03e-06 ***
## model_year71  1.1589904  0.8566572   1.353 0.176887    
## model_year72  0.1278785  0.8515633   0.150 0.880711    
## model_year73 -0.4560456  0.7700023  -0.592 0.554026    
## model_year74  1.7148254  0.8994337   1.907 0.057333 .  
## model_year75  1.0814141  0.8888509   1.217 0.224497    
## model_year76  1.8391012  0.8533677   2.155 0.031782 *  
## model_year77  3.3448755  0.8730736   3.831 0.000149 ***
## model_year78  3.1586630  0.8303347   3.804 0.000166 ***
## model_year79  5.4653256  0.8751153   6.245 1.14e-09 ***
## model_year80  9.5944171  0.9059528  10.590  < 2e-16 ***
## model_year81  7.1203699  0.9054558   7.864 3.92e-14 ***
## model_year82  8.6522401  0.8895808   9.726  < 2e-16 ***
## displacement  0.0157439  0.0054468   2.890 0.004068 ** 
## acceleration  0.0650602  0.0887770   0.733 0.464103    
## ---
## Signif. codes:  0 '***' 0.001 '**' 0.01 '*' 0.05 '.' 0.1 ' ' 1
## 
## Residual standard error: 3.052 on 379 degrees of freedom
## Multiple R-squared:  0.8544, Adjusted R-squared:  0.8475 
## F-statistic: 123.6 on 18 and 379 DF,  p-value: < 2.2e-16
\end{verbatim}

Jak widać acceleration i horsepower są statystycznie nieistotne

Wykresy dla modelu

\begin{Shaded}
\begin{Highlighting}[]
\KeywordTok{plot}\NormalTok{(model)}
\end{Highlighting}
\end{Shaded}

\includegraphics{ProjektMS_files/figure-latex/unnamed-chunk-40-1.pdf}
\includegraphics{ProjektMS_files/figure-latex/unnamed-chunk-40-2.pdf}
\includegraphics{ProjektMS_files/figure-latex/unnamed-chunk-40-3.pdf}
\includegraphics{ProjektMS_files/figure-latex/unnamed-chunk-40-4.pdf}

Obliczenie błędu RMS

\begin{Shaded}
\begin{Highlighting}[]
\NormalTok{predictions <-}\StringTok{ }\KeywordTok{predict}\NormalTok{(model, }\DataTypeTok{newdata =}\NormalTok{ testData)}
\KeywordTok{sqrt}\NormalTok{(}\KeywordTok{mean}\NormalTok{((predictions }\OperatorTok{-}\StringTok{ }\NormalTok{testData}\OperatorTok{$}\NormalTok{mpg)}\OperatorTok{^}\DecValTok{2}\NormalTok{))}
\end{Highlighting}
\end{Shaded}

\begin{verbatim}
## [1] 2.875063
\end{verbatim}

Wykonujemy regresje liniową dla trainingowego zbioru

\begin{Shaded}
\begin{Highlighting}[]
\NormalTok{regresja <-}\StringTok{ }\KeywordTok{lm}\NormalTok{( mpg }\OperatorTok{~}\StringTok{ }\NormalTok{cylinders }\OperatorTok{+}\StringTok{ }\NormalTok{displacement }\OperatorTok{+}\StringTok{ }\NormalTok{horsepower }\OperatorTok{+}\StringTok{ }\NormalTok{weight  }\OperatorTok{+}\StringTok{ }\NormalTok{acceleration }\OperatorTok{+}\StringTok{ }\NormalTok{origin, }\DataTypeTok{data =}\NormalTok{ trainData)}
\KeywordTok{summary}\NormalTok{(regresja)}
\end{Highlighting}
\end{Shaded}

\begin{verbatim}
## 
## Call:
## lm(formula = mpg ~ cylinders + displacement + horsepower + weight + 
##     acceleration + origin, data = trainData)
## 
## Residuals:
##     Min      1Q  Median      3Q     Max 
## -8.5063 -2.1565 -0.4551  1.7093 16.3808 
## 
## Coefficients:
##                Estimate Std. Error t value Pr(>|t|)    
## (Intercept)  36.6666154  3.6173209  10.136  < 2e-16 ***
## cylinders4    9.8190941  2.1472168   4.573 7.37e-06 ***
## cylinders5   12.6864736  4.7215995   2.687 0.007664 ** 
## cylinders6    7.0924165  2.5113052   2.824 0.005097 ** 
## cylinders8    9.5408275  3.0014891   3.179 0.001654 ** 
## displacement -0.0059951  0.0115048  -0.521 0.602728    
## horsepower   -0.0754826  0.0204260  -3.695 0.000266 ***
## weight       -0.0036334  0.0009308  -3.903 0.000120 ***
## acceleration -0.1881372  0.1429827  -1.316 0.189369    
## origin2      -0.4362829  0.8260495  -0.528 0.597830    
## origin3       3.0172575  0.7669529   3.934 0.000107 ***
## ---
## Signif. codes:  0 '***' 0.001 '**' 0.01 '*' 0.05 '.' 0.1 ' ' 1
## 
## Residual standard error: 3.917 on 267 degrees of freedom
## Multiple R-squared:  0.7557, Adjusted R-squared:  0.7466 
## F-statistic: 82.61 on 10 and 267 DF,  p-value: < 2.2e-16
\end{verbatim}

\begin{Shaded}
\begin{Highlighting}[]
\KeywordTok{plot}\NormalTok{(regresja)}
\end{Highlighting}
\end{Shaded}

\includegraphics{ProjektMS_files/figure-latex/unnamed-chunk-42-1.pdf}
\includegraphics{ProjektMS_files/figure-latex/unnamed-chunk-42-2.pdf}
\includegraphics{ProjektMS_files/figure-latex/unnamed-chunk-42-3.pdf}

\begin{verbatim}
## Warning in sqrt(crit * p * (1 - hh)/hh): wyprodukowano wartości NaN

## Warning in sqrt(crit * p * (1 - hh)/hh): wyprodukowano wartości NaN
\end{verbatim}

\includegraphics{ProjektMS_files/figure-latex/unnamed-chunk-42-4.pdf}
displacement i acceleration są nieistotne Ponówny więc regresję

\begin{Shaded}
\begin{Highlighting}[]
\NormalTok{regresja2<-}\StringTok{ }\KeywordTok{lm}\NormalTok{(}\DataTypeTok{formula =}\NormalTok{ mpg }\OperatorTok{~}\StringTok{ }\NormalTok{cylinders }\OperatorTok{+}\StringTok{ }\NormalTok{horsepower }\OperatorTok{+}\StringTok{ }\NormalTok{weight, }\DataTypeTok{data =}\NormalTok{ trainData)}
\KeywordTok{summary}\NormalTok{(regresja2)}
\end{Highlighting}
\end{Shaded}

\begin{verbatim}
## 
## Call:
## lm(formula = mpg ~ cylinders + horsepower + weight, data = trainData)
## 
## Residuals:
##    Min     1Q Median     3Q    Max 
## -9.989 -2.535 -0.500  1.975 15.851 
## 
## Coefficients:
##               Estimate Std. Error t value Pr(>|t|)    
## (Intercept) 38.7385766  2.6930243  14.385  < 2e-16 ***
## cylinders4   7.1210435  2.0872332   3.412 0.000744 ***
## cylinders5   9.2094741  4.6918597   1.963 0.050685 .  
## cylinders6   3.8441012  2.1944651   1.752 0.080952 .  
## cylinders8   6.5123131  2.4053347   2.707 0.007211 ** 
## horsepower  -0.0611112  0.0156619  -3.902 0.000120 ***
## weight      -0.0050545  0.0007904  -6.395 7.02e-10 ***
## ---
## Signif. codes:  0 '***' 0.001 '**' 0.01 '*' 0.05 '.' 0.1 ' ' 1
## 
## Residual standard error: 4.078 on 271 degrees of freedom
## Multiple R-squared:  0.7313, Adjusted R-squared:  0.7254 
## F-statistic: 122.9 on 6 and 271 DF,  p-value: < 2.2e-16
\end{verbatim}

\begin{Shaded}
\begin{Highlighting}[]
\KeywordTok{plot}\NormalTok{(regresja2)}
\end{Highlighting}
\end{Shaded}

\includegraphics{ProjektMS_files/figure-latex/unnamed-chunk-43-1.pdf}
\includegraphics{ProjektMS_files/figure-latex/unnamed-chunk-43-2.pdf}
\includegraphics{ProjektMS_files/figure-latex/unnamed-chunk-43-3.pdf}

\begin{verbatim}
## Warning in sqrt(crit * p * (1 - hh)/hh): wyprodukowano wartości NaN

## Warning in sqrt(crit * p * (1 - hh)/hh): wyprodukowano wartości NaN
\end{verbatim}

\includegraphics{ProjektMS_files/figure-latex/unnamed-chunk-43-4.pdf}
cylinder 6 jest statystycznie nieistotne, ale należy zostawić tą zmienną
z powodu na inne cylindry Tworzymy ramkę danych dla zobaczenia wartości

\begin{Shaded}
\begin{Highlighting}[]
\NormalTok{predykcja <-}\StringTok{ }\KeywordTok{predict}\NormalTok{(regresja2, }\DataTypeTok{newdata =}\NormalTok{ testData)}
\end{Highlighting}
\end{Shaded}

Różnica procentowa dla testowego zbioru

\begin{Shaded}
\begin{Highlighting}[]
\NormalTok{wynik <-}\StringTok{ }\KeywordTok{data.frame}\NormalTok{(}\DataTypeTok{model_year =}\NormalTok{ testData}\OperatorTok{$}\NormalTok{model_year,  }\DataTypeTok{prediction =}\NormalTok{ predykcja,  }\DataTypeTok{actual =}\NormalTok{ testData}\OperatorTok{$}\NormalTok{mpg)}
\NormalTok{roznicaproc <-}\StringTok{ }\KeywordTok{abs}\NormalTok{(wynik}\OperatorTok{$}\NormalTok{prediction }\OperatorTok{-}\StringTok{ }\NormalTok{wynik}\OperatorTok{$}\NormalTok{actual) }\OperatorTok{/}\StringTok{ }
\StringTok{  }\NormalTok{wynik}\OperatorTok{$}\NormalTok{actual }\OperatorTok{*}\StringTok{ }\DecValTok{100}
\NormalTok{wynik}\OperatorTok{$}\NormalTok{roznicaproc <-}\StringTok{ }\NormalTok{roznicaproc}
\KeywordTok{remove}\NormalTok{(roznicaproc)}
\KeywordTok{paste}\NormalTok{(}\StringTok{"Percent difference:"}\NormalTok{, }\KeywordTok{round}\NormalTok{(}\KeywordTok{mean}\NormalTok{(wynik}\OperatorTok{$}\NormalTok{roznicaproc)))}
\end{Highlighting}
\end{Shaded}

\begin{verbatim}
## [1] "Percent difference: 14"
\end{verbatim}

\begin{Shaded}
\begin{Highlighting}[]
\NormalTok{wynik}\OperatorTok{$}\NormalTok{prediction <-}\StringTok{ }\KeywordTok{round}\NormalTok{(wynik}\OperatorTok{$}\NormalTok{prediction, }\DecValTok{2}\NormalTok{)}
\NormalTok{wynik}\OperatorTok{$}\NormalTok{roznicaproc <-}\StringTok{ }\KeywordTok{round}\NormalTok{(wynik}\OperatorTok{$}\NormalTok{roznicaproc, }\DecValTok{2}\NormalTok{)}
\KeywordTok{print}\NormalTok{(wynik)}
\end{Highlighting}
\end{Shaded}

\begin{verbatim}
##     model_year prediction actual roznicaproc
## 1           70      11.21   15.0       25.27
## 2           70      10.32   14.0       26.31
## 3           70       9.13   14.0       34.75
## 4           70      14.18   15.0        5.47
## 5           70      17.07   15.0       13.83
## 6           70      22.63   18.0       25.74
## 7           70      24.31   21.0       15.77
## 8           70      27.04   25.0        8.15
## 9           70      28.08   24.0       16.99
## 10          70      27.66   26.0        6.39
## 11          70      10.91   10.0        9.10
## 12          70       9.54    9.0        5.98
## 13          71      29.72   27.0       10.06
## 14          71      28.79   25.0       15.17
## 15          71      23.16   19.0       21.88
## 16          71      18.78   16.0       17.40
## 17          71       8.58   13.0       34.03
## 18          71      21.34   18.0       18.55
## 19          71      29.38   23.0       27.75
## 20          71      29.63   28.0        5.82
## 21          71      31.10   30.0        3.66
## 22          71      32.93   31.0        6.21
## 23          71      31.70   26.0       21.92
## 24          72      30.22   25.0       20.90
## 25          72      31.17   23.0       35.51
## 26          72      29.35   21.0       39.77
## 27          72      15.18   15.0        1.22
## 28          72      15.03   14.0        7.36
## 29          72      17.52   17.0        3.08
## 30          72      12.95   12.0        7.92
## 31          72      14.99   13.0       15.32
## 32          72      25.49   21.0       21.36
## 33          72      30.58   26.0       17.61
## 34          72      28.50   22.0       29.54
## 35          73      13.83   13.0        6.41
## 36          73      16.45   14.0       17.49
## 37          73      16.99   15.0       13.29
## 38          73      13.52   13.0        4.01
## 39          73      14.67   14.0        4.77
## 40          73       6.48   12.0       46.03
## 41          73      15.24   13.0       17.25
## 42          73      21.94   18.0       21.86
## 43          73      10.83   11.0        1.58
## 44          73      11.34   13.0       12.78
## 45          73      28.96   20.0       44.81
## 46          73       9.57   16.0       40.17
## 47          73      18.90   15.0       26.03
## 48          73      25.69   24.0        7.05
## 49          73      20.94   20.0        4.70
## 50          74      21.10   20.0        5.49
## 51          74      17.36   16.0        8.50
## 52          74      12.33   13.0        5.13
## 53          74      29.57   29.0        1.97
## 54          74      29.47   26.0       13.34
## 55          74      34.35   31.0       10.80
## 56          74      27.35   24.0       13.96
## 57          74      31.66   31.0        2.12
## 58          75      20.84   15.0       38.90
## 59          75      12.67   14.0        9.52
## 60          75      16.11   17.0        5.22
## 61          75      16.47   16.0        2.93
## 62          75      17.65   18.0        1.97
## 63          75      22.25   20.0       11.24
## 64          75      30.30   29.0        4.49
## 65          75      27.07   24.0       12.78
## 66          75      21.57   18.0       19.85
## 67          75      24.99   22.0       13.57
## 68          76      15.39   17.5       12.05
## 69          76      21.49   22.5        4.49
## 70          76      18.02   20.0        9.91
## 71          76      23.70   20.0       18.52
## 72          76      12.11   16.5       26.59
## 73          76      17.75   13.0       36.50
## 74          77      15.46   15.5        0.23
## 75          77      15.60   15.0        3.98
## 76          77      18.85   20.5        8.03
## 77          77      13.81   15.5       10.90
## 78          77      30.30   33.5        9.55
## 79          77      22.43   22.0        1.94
## 80          78      31.12   39.4       21.02
## 81          78      33.09   36.1        8.32
## 82          78      26.73   25.1        6.51
## 83          78      21.87   20.8        5.15
## 84          78      18.01   18.1        0.48
## 85          78      19.08   19.2        0.64
## 86          78      15.09   17.7       14.77
## 87          78      30.81   30.0        2.71
## 88          78      30.00   30.9        2.90
## 89          78      26.23   23.8       10.23
## 90          78      27.78   23.9       16.22
## 91          78      27.35   20.3       34.73
## 92          78      17.22   16.2        6.29
## 93          78      31.46   31.5        0.12
## 94          79      18.88   20.6        8.36
## 95          79      18.54   17.6        5.34
## 96          79      31.79   31.9        0.34
## 97          79      25.40   27.2        6.63
## 98          79      21.91   26.8       18.26
## 99          79      27.44   33.5       18.09
## 100         80      30.87   32.1        3.84
## 101         80      25.18   24.3        3.62
## 102         80      30.03   34.3       12.44
## 103         80      31.12   43.4       28.29
## 104         80      28.94   36.4       20.49
## 105         80      27.85   35.0       20.44
## 106         81      27.41   26.6        3.04
## 107         81      32.47   39.0       16.74
## 108         81      33.30   35.1        5.14
## 109         81      31.90   37.0       13.77
## 110         81      30.27   33.0        8.28
## 111         81      27.75   34.5       19.57
## 112         81      24.64   28.1       12.30
## 113         81      20.44   24.2       15.54
## 114         81      20.01   26.6       24.79
## 115         82      25.76   24.0        7.32
## 116         82      31.75   31.0        2.41
## 117         82      28.75   36.0       20.15
## 118         82      26.50   27.0        1.84
## 119         82      31.92   44.0       27.46
## 120         82      29.13   32.0        8.98
\end{verbatim}

14 procent różnicy nie jest dobrym wynikiem. Lepszym rozwiązaniem będzie
drzewo decyzyjne

\hypertarget{tworzenie-drzewa-decyzyjnego}{%
\subsection{Tworzenie drzewa
decyzyjnego}\label{tworzenie-drzewa-decyzyjnego}}

\begin{Shaded}
\begin{Highlighting}[]
\KeywordTok{library}\NormalTok{(rpart)}
\NormalTok{regresTREE <-}\StringTok{ }\KeywordTok{rpart}\NormalTok{(}\DataTypeTok{formula =}\NormalTok{ mpg }\OperatorTok{~}\StringTok{ }\NormalTok{., }\DataTypeTok{data =}\NormalTok{ testData)}
\NormalTok{dpred <-}\StringTok{ }\KeywordTok{predict}\NormalTok{(regresTREE , }\DataTypeTok{data =}\NormalTok{ testData)}

\KeywordTok{plot}\NormalTok{(regresTREE, }\DataTypeTok{uniform=}\OtherTok{TRUE}\NormalTok{, }\DataTypeTok{main=}\StringTok{"Drzewo decyzyjne"}\NormalTok{)}
\KeywordTok{text}\NormalTok{(regresTREE, }\DataTypeTok{use.n=}\OtherTok{TRUE}\NormalTok{, }\DataTypeTok{all=}\OtherTok{TRUE}\NormalTok{)}
\end{Highlighting}
\end{Shaded}

\begin{verbatim}
## Warning in labels.rpart(x, minlength = minlength): more than 52 levels in a
## predicting factor, truncated for printout
\end{verbatim}

\includegraphics{ProjektMS_files/figure-latex/unnamed-chunk-46-1.pdf}

\begin{Shaded}
\begin{Highlighting}[]
\NormalTok{wynik2 <-}\StringTok{ }\KeywordTok{data.frame}\NormalTok{(}\DataTypeTok{model_year =}\NormalTok{ testData}\OperatorTok{$}\NormalTok{model_year, }
                     \DataTypeTok{prediction =}\NormalTok{ dpred, }
                     \DataTypeTok{actual =}\NormalTok{ testData}\OperatorTok{$}\NormalTok{mpg)}
\NormalTok{roznicaproc2 <-}\StringTok{ }\KeywordTok{abs}\NormalTok{(wynik2}\OperatorTok{$}\NormalTok{prediction }\OperatorTok{-}\StringTok{ }\NormalTok{wynik2}\OperatorTok{$}\NormalTok{actual) }\OperatorTok{/}\StringTok{ }
\StringTok{  }\NormalTok{wynik2}\OperatorTok{$}\NormalTok{actual }\OperatorTok{*}\StringTok{ }\DecValTok{100}
\NormalTok{wynik2}\OperatorTok{$}\NormalTok{roznicaproc2 <-}\StringTok{ }\NormalTok{roznicaproc2}
\KeywordTok{remove}\NormalTok{(roznicaproc2)}
\KeywordTok{paste}\NormalTok{(}\StringTok{"Percent difference:"}\NormalTok{, }\KeywordTok{round}\NormalTok{(}\KeywordTok{mean}\NormalTok{(wynik2}\OperatorTok{$}\NormalTok{roznicaproc2)))}
\end{Highlighting}
\end{Shaded}

\begin{verbatim}
## [1] "Percent difference: 6"
\end{verbatim}

aktualnie błąd jest równy tylko 6 \%

\hypertarget{dendogram}{%
\subsection{Dendogram}\label{dendogram}}

Ponowne losowanie ziarna i losowanie zbioru treningowego i testowego

\begin{Shaded}
\begin{Highlighting}[]
\KeywordTok{set.seed}\NormalTok{(}\DecValTok{100}\NormalTok{)}
\NormalTok{indexes <-}\StringTok{ }\KeywordTok{sample}\NormalTok{(}\KeywordTok{nrow}\NormalTok{(Autko), (}\FloatTok{0.9}\OperatorTok{*}\KeywordTok{nrow}\NormalTok{(Autko)), }\DataTypeTok{replace =} \OtherTok{FALSE}\NormalTok{)}
\NormalTok{trainData <-}\StringTok{ }\NormalTok{Autko[indexes, ]}
\NormalTok{testData <-}\StringTok{ }\NormalTok{Autko[}\OperatorTok{-}\NormalTok{indexes, ]}
\KeywordTok{theme_set}\NormalTok{(}\KeywordTok{theme_bw}\NormalTok{())}
\end{Highlighting}
\end{Shaded}

Wyliczenie średnich i odchyleń, ustalenie dystansów

\begin{Shaded}
\begin{Highlighting}[]
\NormalTok{Autko2<-}\StringTok{ }\NormalTok{testData[,}\OperatorTok{-}\KeywordTok{c}\NormalTok{(}\DecValTok{2}\NormalTok{,}\DecValTok{7}\NormalTok{,}\DecValTok{8}\NormalTok{,}\DecValTok{9}\NormalTok{,}\DecValTok{10}\NormalTok{)]}
\NormalTok{mean_data <-}\StringTok{ }\KeywordTok{apply}\NormalTok{(Autko2,}\DecValTok{2}\NormalTok{,mean)}
\NormalTok{std<-}\StringTok{ }\KeywordTok{apply}\NormalTok{(Autko2, }\DecValTok{2}\NormalTok{,sd)}
\CommentTok{#można to robić manualnie  (x - mean(x)) / sd(x) ale lepsze jest scaling}
\NormalTok{Autko2<-}\KeywordTok{scale}\NormalTok{(Autko2, mean_data, std)}
\NormalTok{distance <-}\StringTok{ }\KeywordTok{dist}\NormalTok{(Autko2)}
\NormalTok{hc<-}\KeywordTok{hclust}\NormalTok{(distance)}
\end{Highlighting}
\end{Shaded}

Przedstawienie na wykresie

\begin{Shaded}
\begin{Highlighting}[]
\KeywordTok{plot}\NormalTok{(hc, }\DataTypeTok{labels =}\NormalTok{ testData}\OperatorTok{$}\NormalTok{origin)}
\CommentTok{#zakładamy 3 klastry, ponieważ są trzy wartości}
\NormalTok{groups <-}\StringTok{ }\KeywordTok{cutree}\NormalTok{(hc, }\DataTypeTok{k=}\DecValTok{3}\NormalTok{)}
\CommentTok{#tworzenie borderów}
\KeywordTok{rect.hclust}\NormalTok{(hc, }\DataTypeTok{k=}\DecValTok{3}\NormalTok{, }\DataTypeTok{border=}\StringTok{"red"}\NormalTok{)}
\end{Highlighting}
\end{Shaded}

\includegraphics{ProjektMS_files/figure-latex/unnamed-chunk-49-1.pdf}

\hypertarget{metoda-k---ux15brednich}{%
\subsection{Metoda K - średnich}\label{metoda-k---ux15brednich}}

Metoda k-średnich jest metodą należacą do grupy algorytmów analizy
skupień tj. analizy polegającej na szukaniu i wyodrębnianiu grup
obiektów podobnych (skupień).

Wykres służący z zapoznaniem się rozmieszczenia originu

\begin{Shaded}
\begin{Highlighting}[]
\KeywordTok{ggplot}\NormalTok{(testData, }\KeywordTok{aes}\NormalTok{(}\DataTypeTok{x=}\NormalTok{ horsepower, }\DataTypeTok{y=}\NormalTok{ displacement, }\DataTypeTok{color =}\NormalTok{ origin)) }\OperatorTok{+}\StringTok{ }\KeywordTok{geom_point}\NormalTok{()}
\end{Highlighting}
\end{Shaded}

\includegraphics{ProjektMS_files/figure-latex/unnamed-chunk-50-1.pdf}

Wybranie odpowiednich kolumn i tworzenie klastrów

\begin{Shaded}
\begin{Highlighting}[]
\NormalTok{testData<-testData }\OperatorTok\StringTok{ }\KeywordTok{select}\NormalTok{(mpg,displacement, horsepower, weight, acceleration, model_year, origin) }
\CommentTok{#wiemy, że mają być 3 klastry }
\NormalTok{k.cluster <-}\StringTok{ }\KeywordTok{kmeans}\NormalTok{(testData[,}\KeywordTok{c}\NormalTok{(}\DecValTok{1}\OperatorTok{:}\DecValTok{6}\NormalTok{)],}\DecValTok{3}\NormalTok{, }\DataTypeTok{nstart =} \DecValTok{20}\NormalTok{)}
\CommentTok{#Wypisanie środków klastrów}
\KeywordTok{print}\NormalTok{(k.cluster}\OperatorTok{$}\NormalTok{centers)}
\end{Highlighting}
\end{Shaded}

\begin{verbatim}
##        mpg displacement horsepower   weight acceleration model_year
## 1 15.24118    326.41176  150.05882 4085.235     13.48235   73.82353
## 2 31.67857     95.92857   75.81924 2120.071     16.87143   77.07143
## 3 21.58889    186.11111   93.00000 3001.000     16.01111   75.11111
\end{verbatim}

Przedstawienie klastrów

\begin{Shaded}
\begin{Highlighting}[]
\KeywordTok{table}\NormalTok{(testData}\OperatorTok{$}\NormalTok{origin, k.cluster}\OperatorTok{$}\NormalTok{cluster)}
\end{Highlighting}
\end{Shaded}

\begin{verbatim}
##    
##      1  2  3
##   1 17  4  6
##   2  0  7  3
##   3  0  3  0
\end{verbatim}

Pierwszy klaster jest dobrze dopasowany, w drugim pojawia się szum
chociaż jest zdecydowanie lepszy od trzeciego

Wykres klastrów

\begin{Shaded}
\begin{Highlighting}[]
\KeywordTok{library}\NormalTok{(cluster) }
\KeywordTok{clusplot}\NormalTok{(testData, k.cluster}\OperatorTok{$}\NormalTok{cluster, }\DataTypeTok{color=}\OtherTok{TRUE}\NormalTok{, }\DataTypeTok{shade=}\OtherTok{TRUE}\NormalTok{, }\DataTypeTok{labels=}\DecValTok{0}\NormalTok{,}\DataTypeTok{lines=}\DecValTok{0}\NormalTok{)}
\end{Highlighting}
\end{Shaded}

\includegraphics{ProjektMS_files/figure-latex/unnamed-chunk-53-1.pdf}
Pokrycie wynosi tylko 81,13\%

\hypertarget{wykresy-obrazujux105ce-analizux119-zbioru}{%
\subsection{Wykresy obrazujące analizę
zbioru}\label{wykresy-obrazujux105ce-analizux119-zbioru}}

Wyliczenie najpopularniejszych marek, obliczenie średniej i wyłuskanie
wartości

\begin{Shaded}
\begin{Highlighting}[]
\NormalTok{namesOccurence <-}\StringTok{ }\NormalTok{Autko }\OperatorTok\StringTok{ }\KeywordTok{group_by}\NormalTok{(brand) }\OperatorTok\StringTok{ }\KeywordTok{tally}\NormalTok{() }\OperatorTok\StringTok{ }\KeywordTok{rename}\NormalTok{(}\DataTypeTok{Number_of_Occurences =}\NormalTok{ n)}
\NormalTok{srednia<-namesOccurence }\OperatorTok\StringTok{ }\KeywordTok{group_by}\NormalTok{(brand) }\OperatorTok\StringTok{ }\KeywordTok{summarise}\NormalTok{(}\DataTypeTok{srednia =} \KeywordTok{mean}\NormalTok{(namesOccurence}\OperatorTok{$}\NormalTok{Number_of_Occurences))}
\NormalTok{srednia<-}\KeywordTok{as.numeric}\NormalTok{(srednia[}\DecValTok{1}\NormalTok{,}\DecValTok{2}\NormalTok{])}
\end{Highlighting}
\end{Shaded}

Wykres najpopularniejszych samochodów (top 20), czarną linią przerywaną
zaznaczono średnią wystąpień

\begin{Shaded}
\begin{Highlighting}[]
\NormalTok{namesOccurence  }\OperatorTok\StringTok{ }\KeywordTok{top_n}\NormalTok{(}\DecValTok{20}\NormalTok{, Number_of_Occurences) }\OperatorTok\StringTok{ }\KeywordTok{arrange}\NormalTok{(}\KeywordTok{desc}\NormalTok{(Number_of_Occurences)) }\OperatorTok\StringTok{ }\KeywordTok{ggplot}\NormalTok{(}\KeywordTok{aes}\NormalTok{(}\DataTypeTok{x=}\NormalTok{brand, }\DataTypeTok{y=}\NormalTok{Number_of_Occurences)) }\OperatorTok{+}
\StringTok{  }\KeywordTok{geom_bar}\NormalTok{(}\DataTypeTok{stat=}\StringTok{'identity'}\NormalTok{) }\OperatorTok{+}
\StringTok{  }\KeywordTok{coord_flip}\NormalTok{() }\OperatorTok{+}\StringTok{  }
\StringTok{  }\KeywordTok{ggtitle}\NormalTok{(}\StringTok{"Najpopularniejsze samochody"}\NormalTok{)}\OperatorTok{+}
\StringTok{  }\KeywordTok{xlab}\NormalTok{(}\StringTok{"Marki samochodów")+}
\StringTok{  ylab("}\NormalTok{Ilość}\StringTok{") + }
\StringTok{  theme_test() +}
\StringTok{  geom_hline(yintercept = srednia, color = "}\NormalTok{black}\StringTok{",linetype=4)+}
\StringTok{  theme(plot.title = element_text(size = 15,  face= 'bold', margin = ))+}
\StringTok{  theme(legend.title =element_text(size = 40, face= 'bold'), legend.position = "}\NormalTok{bottom}\StringTok{")+}
\StringTok{  theme(axis.title.x = element_text( face="}\NormalTok{bold}\StringTok{"))+}
\StringTok{  theme(axis.title.y= element_text( face="}\NormalTok{bold}\StringTok{")) }
\end{Highlighting}
\end{Shaded}

\includegraphics{ProjektMS_files/figure-latex/unnamed-chunk-55-1.pdf}

Najpopularniejszymi markami jest Ford i Chevrolet

Przedstawienie na wykresie kołowym

\begin{Shaded}
\begin{Highlighting}[]
\CommentTok{#wyliczenie procentów}
\NormalTok{namesOccurence}\OperatorTok{$}\NormalTok{procent <-}\StringTok{ }\KeywordTok{round}\NormalTok{(namesOccurence}\OperatorTok{$}\NormalTok{Number_of_Occurences }\OperatorTok{/}\StringTok{ }\KeywordTok{sum}\NormalTok{(namesOccurence}\OperatorTok{$}\NormalTok{Number_of_Occurences), }\DataTypeTok{digits =} \DecValTok{2}\NormalTok{)}
\NormalTok{pie <-}\StringTok{ }\NormalTok{namesOccurence }\OperatorTok\StringTok{ }\KeywordTok{filter}\NormalTok{(procent}\OperatorTok{>}\DecValTok{0}\NormalTok{) }\OperatorTok
\StringTok{  }\KeywordTok{ggplot}\NormalTok{(}\KeywordTok{aes}\NormalTok{(}\DataTypeTok{x =} \StringTok{""}\NormalTok{, }\DataTypeTok{y=}\NormalTok{procent ,}\DataTypeTok{fill =} \KeywordTok{factor}\NormalTok{(brand))) }\OperatorTok{+}\StringTok{ }
\StringTok{  }\KeywordTok{geom_bar}\NormalTok{(}\DataTypeTok{width =} \DecValTok{1}\NormalTok{, }\DataTypeTok{stat =} \StringTok{"identity"}\NormalTok{) }\OperatorTok{+}
\StringTok{  }\KeywordTok{theme}\NormalTok{(}\DataTypeTok{axis.line =} \KeywordTok{element_blank}\NormalTok{(), }
        \DataTypeTok{plot.title =} \KeywordTok{element_text}\NormalTok{(}\DataTypeTok{hjust=}\FloatTok{0.5}\NormalTok{)) }\OperatorTok{+}\StringTok{ }
\StringTok{  }\KeywordTok{labs}\NormalTok{(}\DataTypeTok{fill=}\StringTok{"marki"}\NormalTok{, }
       \DataTypeTok{x=}\OtherTok{NULL}\NormalTok{, }
       \DataTypeTok{y=}\OtherTok{NULL}\NormalTok{, }
       \DataTypeTok{title=}\StringTok{"Wykres kołowy dla marek samochodowych"}\NormalTok{)}

\NormalTok{pie }\OperatorTok{+}\StringTok{ }\KeywordTok{coord_polar}\NormalTok{(}\DataTypeTok{theta =} \StringTok{"y"}\NormalTok{, }\DataTypeTok{start=}\DecValTok{0}\NormalTok{) }\OperatorTok{+}\StringTok{  }\KeywordTok{geom_text}\NormalTok{(}\KeywordTok{aes}\NormalTok{(}\DataTypeTok{x =} \FloatTok{1.3}\NormalTok{, }\DataTypeTok{label =}\NormalTok{ procent), }\DataTypeTok{position =} \KeywordTok{position_stack}\NormalTok{(}\DataTypeTok{vjust =} \FloatTok{0.5}\NormalTok{), }\DataTypeTok{size=}\DecValTok{2}\NormalTok{) }
\end{Highlighting}
\end{Shaded}

\includegraphics{ProjektMS_files/figure-latex/unnamed-chunk-56-1.pdf}

Wykres MPG dla każdej marki

\begin{Shaded}
\begin{Highlighting}[]
\NormalTok{Autko }\OperatorTok\StringTok{ }\KeywordTok{group_by}\NormalTok{(brand) }\OperatorTok\StringTok{ }
\StringTok{  }\KeywordTok{summarise}\NormalTok{(}\DataTypeTok{sredniam =} \KeywordTok{mean}\NormalTok{(mpg, }\DataTypeTok{na.rm =} \OtherTok{TRUE}\NormalTok{))  }\OperatorTok
\StringTok{  }\KeywordTok{ggplot}\NormalTok{(}\KeywordTok{aes}\NormalTok{(}\DataTypeTok{x=}\NormalTok{brand, }\DataTypeTok{y=}\NormalTok{sredniam))}\OperatorTok{+}\KeywordTok{geom_bar}\NormalTok{(}\DataTypeTok{stat=}\StringTok{'identity'}\NormalTok{)  }\OperatorTok{+}\StringTok{ }\KeywordTok{coord_flip}\NormalTok{()}\OperatorTok{+}
\StringTok{  }\KeywordTok{xlab}\NormalTok{(}\StringTok{"Marki samochodów") +ylab("}\NormalTok{Średnia mpg}\StringTok{")+}
\StringTok{  ggtitle("}\NormalTok{Średnia MPG}\StringTok{")+}
\StringTok{  theme(plot.title = element_text(size = 15,  face= 'bold', margin = ))+}
\StringTok{  theme(legend.title =element_text(size = 40, face= 'bold'), legend.position = "}\NormalTok{bottom}\StringTok{")+}
\StringTok{  theme(axis.title.x = element_text( face="}\NormalTok{bold}\StringTok{"))+}
\StringTok{  theme(axis.title.y= element_text( face="}\NormalTok{bold}\StringTok{"))}
\end{Highlighting}
\end{Shaded}

\includegraphics{ProjektMS_files/figure-latex/unnamed-chunk-57-1.pdf}

Wykres ilości aut w danym roku linią przerywaną zaznaczono średnią

\begin{Shaded}
\begin{Highlighting}[]
\NormalTok{df<-Autko}\OperatorTok\StringTok{ }\KeywordTok{group_by}\NormalTok{(model_year) }\OperatorTok\KeywordTok{summarise}\NormalTok{(}\DataTypeTok{ilosc =} \KeywordTok{n}\NormalTok{()) }\OperatorTok\StringTok{ }\KeywordTok{mutate}\NormalTok{(}\DataTypeTok{srednia =} \KeywordTok{mean}\NormalTok{(ilosc))}
\NormalTok{Autko}\OperatorTok\StringTok{ }\KeywordTok{group_by}\NormalTok{(model_year) }\OperatorTok\KeywordTok{summarise}\NormalTok{(}\DataTypeTok{ilosc =} \KeywordTok{n}\NormalTok{()) }\OperatorTok\StringTok{ }\KeywordTok{ggplot}\NormalTok{(}\KeywordTok{aes}\NormalTok{(}\DataTypeTok{x=}\NormalTok{model_year,}\DataTypeTok{y=}\NormalTok{ ilosc ), }\DataTypeTok{lty=}\DecValTok{5}\NormalTok{)}\OperatorTok{+}
\StringTok{  }\KeywordTok{geom_line}\NormalTok{(}\DataTypeTok{group=}\DecValTok{1}\NormalTok{)}\OperatorTok{+}
\StringTok{  }\KeywordTok{geom_point}\NormalTok{(}\DataTypeTok{size=}\DecValTok{2}\NormalTok{)}\OperatorTok{+}
\StringTok{  }\KeywordTok{theme_bw}\NormalTok{()}\OperatorTok{+}
\StringTok{  }\KeywordTok{theme}\NormalTok{(}\DataTypeTok{legend.position =} \StringTok{"none"}\NormalTok{)}\OperatorTok{+}\StringTok{  }\KeywordTok{xlab}\NormalTok{(}\StringTok{"Rok"}\NormalTok{) }\OperatorTok{+}\KeywordTok{ylab}\NormalTok{(}\StringTok{"Ilość"}\NormalTok{)}\OperatorTok{+}
\StringTok{  }\KeywordTok{geom_hline}\NormalTok{(}\DataTypeTok{yintercept =}\NormalTok{ df}\OperatorTok{$}\NormalTok{srednia, }\DataTypeTok{linetype=}\StringTok{"dotted"}\NormalTok{, }\DataTypeTok{color=}\StringTok{"red"}\NormalTok{, }\DataTypeTok{size=}\DecValTok{2}\NormalTok{)}\OperatorTok{+}
\StringTok{  }\KeywordTok{ggtitle}\NormalTok{(}\StringTok{"Ilość aut w danym roku"}\NormalTok{)}\OperatorTok{+}
\StringTok{  }\KeywordTok{geom_text}\NormalTok{(}\DataTypeTok{label=}\StringTok{""}\NormalTok{) }\OperatorTok{+}
\StringTok{  }\KeywordTok{theme}\NormalTok{(}\DataTypeTok{axis.title.x =} \KeywordTok{element_text}\NormalTok{( }\DataTypeTok{face=}\StringTok{"bold"}\NormalTok{))}\OperatorTok{+}
\StringTok{  }\KeywordTok{theme}\NormalTok{(}\DataTypeTok{axis.title.y=} \KeywordTok{element_text}\NormalTok{( }\DataTypeTok{face=}\StringTok{"bold"}\NormalTok{)) }\OperatorTok{+}
\StringTok{  }\KeywordTok{theme}\NormalTok{(}\DataTypeTok{plot.title =} \KeywordTok{element_text}\NormalTok{(}\DataTypeTok{size =} \DecValTok{15}\NormalTok{,  }\DataTypeTok{face=} \StringTok{'bold'}\NormalTok{ ))}
\end{Highlighting}
\end{Shaded}

\includegraphics{ProjektMS_files/figure-latex/unnamed-chunk-58-1.pdf}

Wykres średnich horsepower i displacement w danym roku

\begin{Shaded}
\begin{Highlighting}[]
\NormalTok{Autko }\OperatorTok\StringTok{ }\KeywordTok{group_by}\NormalTok{(model_year)}\OperatorTok\StringTok{ }\KeywordTok{summarise}\NormalTok{(}\DataTypeTok{srednia =} \KeywordTok{mean}\NormalTok{(horsepower), }\DataTypeTok{srednia2 =} \KeywordTok{mean}\NormalTok{(displacement)) }\OperatorTok
\StringTok{  }\KeywordTok{ggplot}\NormalTok{(}\KeywordTok{aes}\NormalTok{(}\DataTypeTok{x=}\NormalTok{model_year))}\OperatorTok{+}
\StringTok{  }\KeywordTok{geom_line}\NormalTok{(}\KeywordTok{aes}\NormalTok{(}\DataTypeTok{x =}\NormalTok{ model_year, }\DataTypeTok{y=}\NormalTok{srednia, }\DataTypeTok{group =} \DecValTok{1}\NormalTok{),}\DataTypeTok{lty=}\DecValTok{2}\NormalTok{, }\DataTypeTok{size=}\FloatTok{0.8}\NormalTok{)}\OperatorTok{+}
\StringTok{  }\KeywordTok{geom_line}\NormalTok{(}\KeywordTok{aes}\NormalTok{(}\DataTypeTok{x=}\NormalTok{ model_year, }\DataTypeTok{y=}\NormalTok{srednia2, }\DataTypeTok{group =} \DecValTok{1}\NormalTok{),}\DataTypeTok{size=}\FloatTok{1.2}\NormalTok{)}\OperatorTok{+}
\StringTok{  }\KeywordTok{scale_color_gradient}\NormalTok{(}\DataTypeTok{low =} \StringTok{'blue'}\NormalTok{, }\DataTypeTok{high =} \StringTok{'red'}\NormalTok{)}\OperatorTok{+}
\StringTok{  }\KeywordTok{theme_bw}\NormalTok{()}\OperatorTok{+}
\StringTok{  }\KeywordTok{annotate}\NormalTok{(}\DataTypeTok{geom=}\StringTok{"text"}\NormalTok{, }\DataTypeTok{x=}\DecValTok{6}\NormalTok{,}\DataTypeTok{y=}\DecValTok{230}\NormalTok{,}
           \DataTypeTok{label=}\StringTok{"displacement"}\NormalTok{,}\DataTypeTok{color=} \StringTok{"black"}\NormalTok{, }\DataTypeTok{size=}\DecValTok{4}\NormalTok{ )}\OperatorTok{+}
\StringTok{  }\KeywordTok{annotate}\NormalTok{(}\DataTypeTok{geom=}\StringTok{"text"}\NormalTok{, }\DataTypeTok{x=}\DecValTok{3}\NormalTok{,}\DataTypeTok{y=}\DecValTok{150}\NormalTok{,}
           \DataTypeTok{label=}\StringTok{"horsepower"}\NormalTok{,}\DataTypeTok{color=} \StringTok{"black"}\NormalTok{, }\DataTypeTok{size=}\DecValTok{4}\NormalTok{ ) }\OperatorTok{+}
\StringTok{  }\KeywordTok{ggtitle}\NormalTok{(}\StringTok{"Średnia dla horsepower i displacement w danym roku"}\NormalTok{)}\OperatorTok{+}
\StringTok{  }\KeywordTok{theme}\NormalTok{(}\DataTypeTok{legend.position =} \StringTok{"none"}\NormalTok{)}\OperatorTok{+}
\StringTok{  }\KeywordTok{labs}\NormalTok{(}\DataTypeTok{x=}\StringTok{"Rok"}\NormalTok{,}\DataTypeTok{y=}\StringTok{""}\NormalTok{) }\OperatorTok{+}
\StringTok{  }\KeywordTok{theme}\NormalTok{(}\DataTypeTok{axis.title.x =} \KeywordTok{element_text}\NormalTok{( }\DataTypeTok{face=}\StringTok{"bold"}\NormalTok{))}\OperatorTok{+}
\StringTok{  }\KeywordTok{theme}\NormalTok{(}\DataTypeTok{axis.title.y=} \KeywordTok{element_text}\NormalTok{( }\DataTypeTok{face=}\StringTok{"bold"}\NormalTok{)) }\OperatorTok{+}
\StringTok{  }\KeywordTok{theme}\NormalTok{(}\DataTypeTok{plot.title =} \KeywordTok{element_text}\NormalTok{(}\DataTypeTok{size =} \DecValTok{15}\NormalTok{,  }\DataTypeTok{face=} \StringTok{'bold'}\NormalTok{ ))}
\end{Highlighting}
\end{Shaded}

\includegraphics{ProjektMS_files/figure-latex/unnamed-chunk-59-1.pdf}

Z roku na rok można zauważyć spadek obu zmiennych

Wykres przedstawiający średnie mpg dla danych marek

\begin{Shaded}
\begin{Highlighting}[]
\CommentTok{#Dodanie nowej kolumny określającej mpg}
\NormalTok{Autko}\OperatorTok{$}\NormalTok{mpgopt<-}\StringTok{ }\KeywordTok{round}\NormalTok{((Autko}\OperatorTok{$}\NormalTok{mpg }\OperatorTok{-}\StringTok{ }\KeywordTok{mean}\NormalTok{(Autko}\OperatorTok{$}\NormalTok{mpg))}\OperatorTok{/}\KeywordTok{sd}\NormalTok{(Autko}\OperatorTok{$}\NormalTok{mpg), }\DecValTok{2}\NormalTok{)}
\NormalTok{Autko}\OperatorTok{$}\NormalTok{typ <-}\StringTok{ }\KeywordTok{ifelse}\NormalTok{(Autko}\OperatorTok{$}\NormalTok{mpgopt }\OperatorTok{<}\StringTok{ }\DecValTok{0}\NormalTok{, }\StringTok{"pod"}\NormalTok{, }\StringTok{"nad"}\NormalTok{)}
\CommentTok{#Wykres}
\NormalTok{Autko }\OperatorTok\KeywordTok{group_by}\NormalTok{(brand) }\OperatorTok\StringTok{ }\KeywordTok{ggplot}\NormalTok{(}\KeywordTok{aes}\NormalTok{(}\DataTypeTok{x=}\NormalTok{brand, }\DataTypeTok{y=}\NormalTok{mpgopt, }\DataTypeTok{label=}\NormalTok{mpgopt)) }\OperatorTok{+}\StringTok{ }
\StringTok{  }\KeywordTok{geom_bar}\NormalTok{(}\DataTypeTok{stat=}\StringTok{'identity'}\NormalTok{, }\KeywordTok{aes}\NormalTok{(}\DataTypeTok{fill=}\NormalTok{typ), }\DataTypeTok{width=}\NormalTok{.}\DecValTok{5}\NormalTok{)  }\OperatorTok{+}
\StringTok{  }\KeywordTok{scale_fill_manual}\NormalTok{(}\DataTypeTok{name=}\StringTok{"Według mpg"}\NormalTok{, }
                    \DataTypeTok{labels =} \KeywordTok{c}\NormalTok{(}\StringTok{"Powyżej średniej"}\NormalTok{, }\StringTok{"Poniżej średniej"}\NormalTok{), }
                    \DataTypeTok{values =} \KeywordTok{c}\NormalTok{(}\StringTok{"nad"}\NormalTok{=}\StringTok{"#00ba38"}\NormalTok{, }\StringTok{"pod"}\NormalTok{=}\StringTok{"#f8766d"}\NormalTok{)) }\OperatorTok{+}\StringTok{ }
\StringTok{  }\KeywordTok{labs}\NormalTok{( }\DataTypeTok{title=} \StringTok{"Średnie mpg dla danych marek"}\NormalTok{) }\OperatorTok{+}\StringTok{ }
\StringTok{  }\KeywordTok{coord_flip}\NormalTok{() }\OperatorTok{+}\StringTok{ }
\StringTok{  }\KeywordTok{labs}\NormalTok{(}\DataTypeTok{x=}\StringTok{"Marka"}\NormalTok{,}\DataTypeTok{y=}\StringTok{"Średnia mpg"}\NormalTok{)}
\end{Highlighting}
\end{Shaded}

\includegraphics{ProjektMS_files/figure-latex/unnamed-chunk-60-1.pdf}

\end{document}
